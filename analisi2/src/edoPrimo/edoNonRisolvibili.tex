% Imposto la radice del documento, utile per Visual Studio Code ed altri editor
%! TEX root = ../../analisi2.tex

% Imposto il file come sottofile del documento principale
\documentclass[../../analisi2]{subfiles}

\begin{document}

    \chapter{Equazioni differenziali non lineari del primo ordine risolvibili esplicitamente}

        Non tutte le equazione diferenziali non lineari del primo ordine sono risolvibili esplicitamente. Quelle che possono essere
        risolte sono:

        \begin{enumerate}
            \item EDO a variabili separabili,
            \item EDO non lineari omogenee,
            \item Equazioni di Bernoulli.
        \end{enumerate}

        Le equazioni a variabili separabili sono già state trattate nel \hyperref[cap:edoSeparabili]{\textbf{relativo capitolo}}.
        Discutiamo dunque delle EDO non lineari omogenee.

        \section{Equazioni differenziali ordinarie non lineari omogenee}

            Partiamo dalla definizione.

            \begin{definizione}[EDO non lineari omogenee]
                Una EDO non lineare omogenea è della forma
                \[
                    y'(x) = \frac{P(x, \, y(x))}{Q(x, \, y(x))},
                \]
                con \(P\) e \(Q\) polinomi omogenei, ovvero composti solamente da monomi di uno stesso grado \(n\).
            \end{definizione}

            Delineiamo quindi una strategia di risoluzione:
            \begin{enumerate}
                \item Dividiamo ciascun monomio per \(x^n\).
                \item Eseguiamo il cambio di variabile con
                    \[
                        z(x) = \frac{y(x)}{x}.  
                    \]
                \item Otteniamo dunque un'equazione a variabili separabili che possiamo facilmente risolvere.
                \item Infine torniamo alla variabile iniziale \(y\), tenendo a mente che \(y(x) = x \, z(x)\), in base al cambio di
                    variabile precedentemente fatto.
            \end{enumerate}

        \section{Equazioni di Bernoulli}

            L'ultima categoria è quella delle equazioni di Bernoulli. Partiamo anche in questo caso dalla definizione.

            \begin{definizione}[Equazione di Bernoulli]
                Chiamiamo equazione di Bernoulli una EDO non lineare del tipo
                \[
                    y'(x) = f(x) \, y(x) + g(x) \, y(x)^\alpha,
                \]
                con \(f, \, g : \I \subseteq \R \to \R\) continue e \(\alpha \in \R\), con \(\alpha \notin \{0, \, 1\}\).
            \end{definizione}

            Le condizioni su \(\alpha\) sono state imposte in quanto:
            \begin{itemize}
                \item Per \(\alpha = 0\), l'equazione diventa \(y'(x) = f(x) \, y(x) + g(x)\), la quale è lineare;
                \item Per \(\alpha = 1\), l'equazione diventa \(y'(x) = \left(f(x) + g(x)\right) \, y(x)\), la quale è lineare a variabili
                    separabili.
            \end{itemize}

            Delineiamo ora la strategia di risoluzione:
            \begin{enumerate}
                \item Cerchiamo le soluzioni costanti:
                    \begin{itemize}
                        \item Se \(\alpha > 1\) poniamo
                            \[
                                f(x) \, y + g(x) \, y^\alpha = 0,
                            \]
                            che possiamo riscrivere come
                            \[
                                y \left(f(x) + g(x) \, y^{\alpha - 1})\right) = 0,
                            \]
                            che implica
                            \[
                                y(x) = 0 \quad \forall\,  x.
                            \]
                        \item Se \(0 < \alpha < 1\) poniamo
                            \[
                                f(x) \, y + g(x) \, y^\alpha = 0,
                            \]
                            che possiamo riscrivere come
                            \[
                                y^\alpha \left(f(x) \, y^{1 - \alpha} + g(x)\right) = 0,
                            \]
                            che implica
                            \[
                                y(x) = 0 \quad \forall\,  x.
                            \]
                        \item Se \(f\) e \(g\) sono delle costanti in \(\R\), troviamo un'ulteriore soluzione costante come
                            \[
                                y(x) = \left(-\frac{f}{g}\right)^{\frac{1}{\alpha - 1}} \quad \forall \, x.
                            \]
                    \end{itemize}
                \item Per cercare le soluzioni non costanti, dividiamo l'equazione per \(y^\alpha (x)\), ottenendo
                    \[
                        \frac{y'(x)}{y^\alpha (x)} = \frac{f'(x)}{y^{\alpha - 1} (x)} + g(x).
                    \]
                    Dunque l'equazione diventa
                    \[
                        \frac{1}{1 - \alpha} \left(\frac{1}{y^{\alpha - 1} (x)}\right)' = \frac{1}{y^{\alpha - 1} (x)} + g(x).
                    \]
                \item Effetuiamo il cambio di variabili con
                    \[
                        z(x) = \frac{1}{y^{\alpha - 1} (x)}.
                    \]
                    Sostituendo si ottiene
                    \[
                        \frac{1}{1 - \alpha} z'(x) = f(x) \, z(x) + g(x),
                    \]
                    o, più esplicitamente,
                    \[
                        z'(x) - (1 - \alpha) \, f(x) \, z(x) = (1 - \alpha) \, f(x),
                    \]
                    la quale è una EDO lineare con
                    \begin{itemize}
                        \item \(a(x) = - (1 - \alpha) \, g(x)\),
                        \item \(b(x) = (1 - \alpha) \, g(x)\).
                    \end{itemize}
                \item Si risolve l'equazione lineare in \(z\) e poi si torna alla variabile \(y(x)\).
            \end{enumerate}
            
\end{document}