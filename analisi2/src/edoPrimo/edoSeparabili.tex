% Imposto la radice del documento, utile per Visual Studio Code ed altri editor
%! TEX root = ../../analisi2.tex

% Imposto il file come sottofile del documento principale
\documentclass[../../analisi2]{subfiles}

\begin{document}

    \chapter{Equazioni differenziali del primo ordine a variabili separabili}
    \label{cap:edoSeparabili}

        Introduciamo ora il concetto di equazione differenziale ordinaria del primo ordine a variabili separabili.

        \begin{definizione}[EDO a variabili separate]
            Una EDO del primo ordine in forma normale si dice a variabili separabili se è della forma
            \[
                y'(x) = h(x) \; g(y(x)),
            \]
            con \(h : \mathrm{J}_1 \subseteq \R \to \R\) e \(g : \mathrm{J}_2 \subseteq \R \to \R\) continue.

            Cioè \(f(x, \, y) = h(x) \times g(y)\) è il prodotto di una funzione che dipende solo da \(x\) per una funzione che dipende solo
            da \(y\).
        \end{definizione}

        A questo punto passiamo alla loro risoluzione.

        \begin{teorema}[Risoluzione di EDO a variabili separabili]
            Consideriamo la EDO a variabili separate
            \[
                y'(x) = h(x) \; g(y(x)),
            \]
            con \(h : \mathrm{J}_1 \subseteq \R \to \R\) e \(g : \mathrm{J}_2 \subseteq \R \to \R\) continue.

            Possiamo dunque dividere le soluzioni in due categorie:
            \begin{enumerate}
                \item Se \(g(D) = 0\) per qualche \(D \in \R\) allora la funzione costante \(y(x) = D, \, \forall \, x\) è soluzione.
                \item Se \(g(y) \neq 0, \forall \, y\) in un certo intervallo, una soluzione \(y(x)\) è definita implicitamente
                    dall'equazione
                    \[
                        \Gamma(y(x)) = H(x) + \mathrm{c},
                    \]
                    dove
                    \begin{itemize}
                        \item \(\mathrm{c} \in \R\);
                        \item \(H\) è una primitiva di \(h\);
                        \item \(\Gamma\) è una primitiva di \(\frac{1}{g}\).
                    \end{itemize}

                    Se \(\Gamma\) non è invertibile otteniamo eslicitamente
                    \[
                        y(x) = \Gamma^{-1}(H(x) + \mathrm{c}),
                    \]
                    con \(c \in \R\).
            \end{enumerate}
        \end{teorema}
        \newpage
        \begin{proof}
            Dimostriamo entrambe le categorie:
            \begin{enumerate}
                \item Sia \(g(D) = 0\) e \(y(x) = D\), \(\forall \, x\).
                    L'identità è soddisfatta:
                    \begin{enumerate}
                        \item Sinistra: \(y'(x) = 0\);
                        \item Destra: \(h(x) \, g(y(x)) = h(x) \, g(D) = h(x) \times 0 = 0\).
                    \end{enumerate}
                \item Prendiamo un intervallo \([x_0, x]\) in cui la funzione \(g(y)\) non si annulla. Dunque,
                    \[
                        \frac{y'(x)}{g(y(x))} = h(x).
                    \]

                    Dati \(x_0 < x\), con \(x_0, \, x \in \mathrm{J}_1\), integriamo:
                    \[
                        \intDef{x_0}{x}{\frac{y'(r)}{g(y(r))}}{r} = \intDef{x_0}{x}{h(r)}{r} = H(x) + \mathrm{c}.
                    \]

                    Per il lato sinistro faccio il cambio di variabili:
                    \begin{align*}
                        y(r) &= k,\\
                        y'(r) &= \mathrm{d}k,
                    \end{align*}
                    quando
                    \begin{align*}
                        r = x_0 &\Rightarrow k = y(x_0),\\
                        r = x &\Rightarrow k = y(x).
                    \end{align*}

                    Dunque,
                    \[
                        \intDef{x_0}{x}{\frac{y'(r)}{g(y(r))}}{r} = \intDef{y(x_0)}{y(x)}{\frac{1}{g(k)}}{k} = \Gamma(y(x)) + \mathrm{c}.
                    \]

                    Uguagliando i due lati,
                    \[
                        \Gamma(y(x)) = H(x) + \mathrm{c},
                    \]
                    con \(\mathrm{c} \in \R\).
            \end{enumerate}
        \end{proof}
            
\end{document}