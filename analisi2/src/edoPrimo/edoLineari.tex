% Imposto la radice del documento, utile per Visual Studio Code ed altri editor
%! TEX root = ../../analisi2.tex

% Imposto il file come sottofile del documento principale
\documentclass[../../analisi2]{subfiles}

\begin{document}

    \chapter{Equazioni differenziali lineari del primo ordine}

        Introduciamo ora il concetto di equazione differenziale lineare ordinaria del primo ordine.

        \begin{definizione}[EDO lineari del primo ordine]
            Un'equazione differenziale lineare ordinaria del primo ordine è una EDO nella forma
            \[
                c(x) \, y'(x) + a(x) \, y(x) = b(x),
            \]
            con \(c, a, b : \mathrm{J} \subseteq \R \to \R\) continue su \(\mathrm{J}\).

            Ci occuperemo solo di EDO lineari del primo ordine in forma normale, cioè
            \[
                y'(x) + a(x) \, y(x) = b(x).
            \]
        \end{definizione}

        A questo punto passiamo alla loro risoluzione.

        \begin{teorema}[Integrale generale di una EDO lineare del primo ordine]
            Date \(a, b : \mathrm{J} \subseteq \R \to \R\) continue, consideriamo
            \[
                y'(x) + a(x) \, y(x) = b(x).
            \]

            L'integrale generale di questa equazione è dato dalla formula
            \[
                y(x) = e^{-A(x)} \left[B(x) + \mathrm{c}\right],
            \]
            dove
            \begin{itemize}
                \item \(\mathrm{c} \in \R\);
                \item \(A(x)\) è una qualunque primitiva di \(a(x)\): \(A = \int \! a\);
                \item \(B(x)\) è una qualunque primitiva di \(e^{A(x)} \, b(x)\): \(A = \int \! e^{\int \! a} \, b\).
            \end{itemize}
        \end{teorema}
        \begin{proof}
            Moltiplichiamo tutto per \(a^A\)
            \[
                \underbrace{y' \, e^A + a \, y \, e^A}_{\left(y \, e^A\right)'} = b \, e^A,
            \]
            infatti
            \[
                \left(y \, e^A\right)' = y' \, e^A + y \, \left(e^A\right)' = y' \, e^A + y \, e^A \, A' = y' \, e^A + y \, e^A \, a.
            \]
            Quindi la nostra equazione è uguale a
            \[
                \left(y(x) \, e^{A(x)}\right)' = e^{A(x)} \, b(x).
            \]
            Ora integriamo tra \(x_0\) e \(x\):
            \[
                \int_{x_0}^x \! \left(y(s) \, e^{A(s)}\right)' \; \mathrm{d}s = \int_{x_0}^{x} \! e^{A(s)} \, b(s) \; \mathrm{d}s.
            \]
            Uso il teorema fondamentale del calcolo integrale per il lato sinistro:
            \[
                y(x) \, e^{A(x)} \underbrace{- y(x_0) \, e^{A(x_0)}}_{- \mathrm{c}} = \underbrace{\int_{x_0}^x \! e^{A(s)} \, b(s) \; \mathrm{d}s}_{B(x)}.
            \]
            Per ottenere la formula è sufficiente moltiplicare questa equazione per \(e^{-A(x)}\).
        \end{proof}
            
\end{document}