% Imposto la radice del documento, utile per Visual Studio Code ed altri editor
%! TEX root = ../../analisi2.tex

% Imposto il file come sottofile del documento principale
\documentclass[../../analisi2]{subfiles}

\begin{document}

    \chapter{Equazioni differenziali lineari del primo ordine}

        Introduciamo ora il concetto di equazione differenziale lineare ordinaria del primo ordine.

        \begin{definizione}[EDO lineari del primo ordine]
            Un'equazione differenziale lineare ordinaria del primo ordine è una EDO nella forma
            \[
                c(x) \, y'(x) + a(x) \, y(x) = b(x),
            \]
            con \(c, a, b : \mathrm{J} \subseteq \R \to \R\) continue su \(\mathrm{J}\).

            Ci occuperemo solo di EDO lineari del primo ordine in forma normale, cioè
            \[
                y'(x) + a(x) \, y(x) = b(x).
            \]
        \end{definizione}

        A questo punto passiamo alla loro risoluzione.

        \begin{teorema}[Integrale generale di una EDO lineare del primo ordine]
            Date \(a, b : \mathrm{J} \subseteq \R \to \R\) continue, consideriamo
            \[
                y'(x) + a(x) \, y(x) = b(x).
            \]

            L'integrale generale di questa equazione è dato dalla formula
            \[
                y(x) = e^{-A(x)} \left[B(x) + \mathrm{c}\right],
            \]
            dove
            \begin{itemize}
                \item \(\mathrm{c} \in \R\);
                \item \(A(x)\) è una qualunque primitiva di \(a(x)\): \(A = \int \! a\);
                \item \(B(x)\) è una qualunque primitiva di \(e^{A(x)} \, b(x)\): \(A = \int \! e^{\int \! a} \, b\).
            \end{itemize}
        \end{teorema}
        \begin{proof}
            Moltiplichiamo tutto per \(a^A\)
            \[
                \underbrace{y' \, e^A + a \, y \, e^A}_{\left(y \, e^A\right)'} = b \, e^A,
            \]
            infatti
            \[
                \left(y \, e^A\right)' = y' \, e^A + y \, \left(e^A\right)' = y' \, e^A + y \, e^A \, A' = y' \, e^A + y \, e^A \, a.
            \]
            Quindi la nostra equazione è uguale a
            \[
                \left(y(x) \, e^{A(x)}\right)' = e^{A(x)} \, b(x).
            \]
            Ora integriamo tra \(x_0\) e \(x\):
            \[
                \intDef{x_0}{x}{\left(y(s) \, e^{A(s)}\right)'}{s} = \intDef{x_0}{x}{e^{A(s)} \, b(s)}{s}.
            \]
            Uso il teorema fondamentale del calcolo integrale per il lato sinistro:
            \[
                y(x) \, e^{A(x)} \underbrace{- y(x_0) \, e^{A(x_0)}}_{- \mathrm{c}} = \underbrace{\intDef{x_0}{x}{e^{A(s)} \, b(s)}{s}}_{B(x)}.
            \]
            Per ottenere la formula è sufficiente moltiplicare questa equazione per \(e^{-A(x)}\).
        \end{proof}

        Introduciamo ora le EDO omogenee del primo ordine con alcuni importanti teoremi.

        \begin{definizione}[Equazione omogenea associata ad una EDO del primo ordine]
            Data una EDO lineare del primo ordine nella forma
            \[
                y'(x) + a(x) \, y(x) = b(x),
            \]
            si chiama \textbf{equazione omogenea associata} la EDO
            \[
                y'(x) + a(x) \, y(x) = 0.
            \]
        \end{definizione}

        \begin{teorema}[Principio di sovrapposizione per EDO lineari omogenee del primo ordine]
            Data \(a : \mathrm{J} \subseteq \R \to \R\) continua, consideriamo la EDO omogenea
            \[
                y'(x) + a(x) \, y(x) = 0.
            \]
            Se \(y_1\) e \(y_2\) sono due soluzioni di questa equazione e \(\alpha_1, \, \alpha_2 \in \R\), allora
            \[
                \alpha_1 \, y_1(x) + \alpha_2 \, y_2(x)
            \]
            è anch'essa soluzione dell'equazione.
        \end{teorema}
        \begin{proof}
            Sapendo che
            \[
                y_1'(x) + a(x) \, y_1(x) = y_2'(x) + a(x) \, y_2(x) = 0,
            \]
            dobbiamo dimostrare che
            \[
                \left(\alpha_1 \, y_1(x) + \alpha_2 \, y_2(x)\right)' + a(x) \left(\alpha_1 \, y_1(x) + \alpha_2 \, y_2(x)\right) = 0.
            \]
            Possiamo riarrangiare i termini per linearità:
            \begin{gather*}
                \left(\alpha_1 \, y_1(x) + \alpha_2 \, y_2(x)\right)' + a(x) \left(\alpha_1 \, y_1(x) + \alpha_2 \, y_2(x)\right) =\\
                \qquad\qquad
                \begin{aligned}
                    &= \alpha_1 \, y_1'(x) + \alpha_2 \, y_2'(x) + a(x) \, \alpha_1 \, y_1(x) + a(x) \, \alpha_2 \, y_2(x)\\
                    &= \alpha_1 \left(y_1'(x) + a(x) \, y_1(x)\right) + \alpha_2 \left(y_2'(x) + a(x) \, y_2(x)\right)\\
                    &= \alpha_1 \times 0 + \alpha_2 \times 0 = 0
                \end{aligned}
            \end{gather*}
        \end{proof}

        \begin{teorema}[Struttura dell'integrale generale di EDO lineari del primo ordine]
            Siano \(a, \, b : \mathrm{J} \subseteq \R \to \R\) continue:
            \begin{enumerate}
                \item L'integrale generale dell'equazione omogenea
                    \[
                        y'(x) + a(x) \, y(x) = 0
                    \]
                    è uno spazio vettoriale di dimensione unitaria, cioè
                    \[
                        y_O(x) = \mathrm{C} \times \overline{y_O}(x),
                    \]
                    con \(\mathrm{C} \in \R\).
                \item L'integrale generale dell'equazione completa
                    \[
                        y'(x) + a(x) \, y(x) = b(x)
                    \]
                    equivale a
                    \[
                        y(x) = y_O(x) + y_P(x),
                    \]
                    dove
                    \begin{itemize}
                        \item \(y_O(x)\) è l'integrale dell'equazione omogenea, come al punto I;
                        \item \(y_P(x)\) è una soluzione particolare dell'equazione completa.
                    \end{itemize}
            \end{enumerate}
        \end{teorema}

        Introduciamo infine il concetto di problema di Cauchy.

        \begin{definizione}[Problema di Cauchy]
            Data la EDO del primo ordine in forma normale
            \[
                y'(x) = f\left(x, \, y(x)\right)
            \]
            ed assegnata la coppia \((x_0, \, y_0) \in \R^2\) nella quale \(f\) è ben definita, si chiama \textbf{problema di Cauchy}
            il problema di determinare \(y : \mathrm{I} \subseteq \R \to \R\) che soddisfa
            \[
                \left\{
                    \begin{aligned}
                        &y'(x) = f \left(x, \, y(x)\right)\\
                        &y(x_0) = y_0
                    \end{aligned}
                \right.
                .
            \]
        \end{definizione}
            
\end{document}