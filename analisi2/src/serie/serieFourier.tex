% Imposto la radice del documento, utile per Visual Studio Code ed altri editor
%! TEX root = ../../analisi2.tex

% Imposto il file come sottofile del documento principale
\documentclass[../../analisi2]{subfiles}

\begin{document}

    \chapter{Serie di Fourier}

        Partiamo parlando di polinomi trigonometrici ed armoniche \(n\)-esime.

        \begin{definizione}[Funzione periodica]
            Una funzione \(f : \R \to \R\) è periodica di periodo \(T\) se
            \[
                f(x + T) = f(x), \quad \forall \, x \in \R.
            \]
        \end{definizione}

        \begin{definizione}[Polinomio trigonometrico]
            Un \textbf{polinomio trigonometrico di ordine \(m\)}, con \(m \in \N\), è una funzione periodica di periodo \(2 \pi\) del
            tipo
            \[
                \mathrm{P}_m (x) = a_0 + \sum_{n = 1}^m \left(a_n \cos nx + b_n \sin nx\right),
            \]
            con \(a_0, \, a_n, \, b_n \in \R\), detti coefficienti del polinomio, e \(x \in \R\). Le funzioni \(\cos nx\) e \(\sin nx\)
            sono dette \textbf{armoniche \(n\)-esime}.
        \end{definizione}

        Proprietà:
        \begin{enumerate}
            \item Ogni armonica \(n\)-esima è periodica di periodo \(\frac{2 \pi}{n}\), quindi, in particolare, ogni armonica \(n\)-esima
                è anche periodica di periodo \(2\pi\). Dunque, anche ogni polinomio trigonometrico è periodico di periodo \(2\pi\).
            \item La somma, la differenza ed il prodotto di due polinomi trigonometrici è ancora un polinomio trigonometrico.
            \item Formule di ortogonalità delle armoniche \(n\)-esime:
                \begin{align*}
                    \intDef{-\pi}{\pi}{\cos nx \, \cos kx}{x} &=
                        \begin{cases}
                            0 & \text{se } n \neq k\\
                            \pi & \text{se } n = k \neq 0
                        \end{cases}\\
                    \intDef{-\pi}{\pi}{\sin nx \, \sin kx}{x} &=
                        \begin{cases}
                            0 & \text{se } n \neq k\\
                            \pi & \text{se } n = k \neq 0
                        \end{cases}\\
                    \intDef{-\pi}{\pi}{\sin nx \, \cos kx}{x} &= 0, \quad \forall \, n, \, k \in \N.
                \end{align*}
                \begin{osservazione}
                    Si tratta di formule di ortogonalità, ma non specifichiamo qual è il prodotto scalare e qual è lo spazio.
                \end{osservazione}
        \end{enumerate}

        Parliamo dunque delle serie di Fourier di funzioni periodiche. Sappiamo da Analisi Matematica I che una funzione derivabile
        infinite volte è approssimata dal suo polinomio di Taylor. Abbiamo inoltre visto che una funzione analitica coincide con la sua
        serie di Taylor.

        Adesso vogliamo occuparci di approssimare una funzione periodica con un polinomio trigonometrico e stabilire in quale caso la
        funzione coincide con la sua serie di Fourier.

        Data \(f : \R \to \R\) periodica di periodo \(2\pi\), esiste una serie trigonometrica la cui somma sia proprio \(f\)? Cioè
        esistono \(a_0, \, a_, \, b_n\) tali che
        \[
            f(x) = a_0 + \sum_{n = 1}^{+\infty} \left(a_n \cos nx + b_n \sin nx\right)?
        \]

        \begin{osservazione}
            Adesso \(f\) è periodica, il che è molto meno regolare che derivabile infinite volte. Infatti, la funzione può anche essere
            discontinua.
        \end{osservazione}

        Vediamo ora come calcolare i coefficienti di una serie di Fourier.
        
        \begin{teorema}[Calcolo dei coefficienti di una serie di Fourier]
            Data \(f : \R \to \R\) periodica di periodo \(2\pi\), la serie trigonometrica candidata è
            \[
                a_0 + \sum_{n = 1}^{+\infty} \left(a_n \cos nx + b_n \sin nx\right),
            \]
            dove \(n = 1, \, 2, \, 3, \, \ldots\) e
            \begin{align*}
                a_0 &= \frac{1}{2\pi} \intDef{-\pi}{\pi}{f(x)}{x},\\
                a_n &= \frac{1}{\pi} \intDef{-\pi}{\pi}{f(x) \cos nx}{x},\\
                b_n &= \frac{1}{\pi} \intDef{-\pi}{\pi}{f(x) \sin nx}{x},
            \end{align*}
            i quali si ottengono usando le formule di ortogonalità.

            Per calcolare \(a_0\), integriamo la serie candidata tra \(-\pi\) e \(\pi\):
            \begin{align*}
                \intDef{-\pi}{\pi}{f(x)}{x} &= \intDef{-\pi}{\pi}{\left(a_0 + \sum_{n = 1}^{+\infty}\left(a_n \cos nx + b_n \sin nx\right)\right)}{x}\\
                &= \intDef{-\pi}{\pi}{a_0}{x} + \cancelto{0}{\sum_{n = 1}^{+\infty} a_n \intDef{-\pi}{\pi}{\cos nx}{x}} + \cancelto{0}{\sum_{n = 1}^{+\infty} b_n \intDef{-\pi}{\pi}{\sin nx}{x}} \qquad \text{per simmetria}\\
                &= 2 \pi a_0.
            \end{align*}

            Per determinare \(a_k\) moltiplico la serie candidata per \(\cos nx\) ed integro tra \(-\pi\) e \(\pi\):
            \begin{align*}
                \intDef{-\pi}{\pi}{f(x) \cos kx}{x} &= \intDef{-\pi}{\pi}{\left(a_0 + \sum_{n = 1}^{+\infty} \left(a_n \cos nx + b_n \sin nx\right)\right) \cos kx}{x}\\
                &= \cancelto{0}{a_0 \intDef{-\pi}{\pi}{\cos kx}{x}} \; + \underbrace{\sum_{n = 1}^{+\infty} a_n \intDef{-\pi}{\pi}{\cos nx \cos kx}{x}}_{\text{non nullo solo per } n = k} + \cancelto{0}{\sum_{n = 1}^{+\infty} b_n \intDef{-\pi}{\pi}{\sin nx \cos kx}{x}}\\
                &= a_k \intDef{-\pi}{\pi}{\cos^2 kx}{x}\\
                &= \pi a_k.
            \end{align*}
            Analogamente, per ottenere \(b_k\) si moltiplica per \(\sin kx\).
        \end{teorema}
        \begin{osservazione}
            Abbiamo supposto di poter scambiare il simbolo di serie ed il simbolo di integrale.
        \end{osservazione}

\end{document}