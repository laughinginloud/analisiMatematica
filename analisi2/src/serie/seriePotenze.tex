% Imposto la radice del documento, utile per Visual Studio Code ed altri editor
%! TEX root = ../../analisi2.tex

% Imposto il file come sottofile del documento principale
\documentclass[../../analisi2]{subfiles}

\begin{document}

    \chapter{Serie di potenze in \texorpdfstring{\(\R\)}{R}}

        Introduciamo il concetto di serie di potenze.

        \begin{definizione}[Serie di potenze]
            Una serie di potenze reale è una serie di funzioni nella forma
            \[
                \sum_{n = 0}^{+\infty} a_n \, (x - x_0)^n = a_0 + a_1 \, (x - x_0) + a_2 \, (x - x_0)^2 + \ldots + a_n \, (x - x_0)^n +
                \ldots,
            \]
            con \(a_n \in \R\) \textbf{coefficienti} della serie e \(x_0 \in \R\) \textbf{centro} della serie.
        \end{definizione}
        \begin{osservazione}
            Stiamo adottando la convenzione che, nel caso \(x = x_0\) e \(n = 0\), si ha
            \[
                a_0 \, (x_0 - x_0)^0 = a_0 \times 1 = a_0.
            \]
            Si tratta di una convenzione, in quanto la scrittura \(0^0\) in realtà non è determinata.

            In particolare, nel caso \(x = x_0\) e \(n\) generico si ha
            \[
                \sum_{n = 0}^{+\infty} a_n \, (x_0 - x_0)^n = a_0 + a_1 \times 0 + a_2 \times 0 + \ldots = a_0,
            \]
            dunque tutte le serie di potenze convergono in \(x = x_0\).
        \end{osservazione}

        L'insieme di convergenza di una serie di potenze è sempre un intervallo centrato in \(x_0\). Questo ci porta al nostro
        prossimo teorema.

        \begin{teorema}[Raggio di convergenza di una serie di potenze reale]
            Data una serie di potenze reale
            \[
                \sum_{n = 0}^{+\infty} a_n \, (x - x_0)^n,
            \]
            si verifica sempre una tra tre condizioni:
            \begin{enumerate}
                \item la serie converge solo per \(x = x_0\), dunque si ha un raggio di convergenza nullo;
                \item la serie converge assolutamete su tutto \(\R\), dunque si ha un raggio di convergenza infinito;
                \item esiste un numero reale \(\R > 0\) tale che:
                    \begin{itemize}
                        \item la serie converge assolutamente per ogni \(x\) con \(|x - x_0|  < R\),
                        \item la serie non converge per \(|x - x_0| > R\),
                    \end{itemize}
                    dunque si ha un raggio di convergenza \(R\).
            \end{enumerate}
        \end{teorema}
        \begin{osservazione}
            Per quanto riguarda il caso numero tre, nulla è detto riguardo ai punti \(x = x_0 \pm R\), perciò vanno analizzati caso
            per caso.
        \end{osservazione}
        \begin{proof}
            Dobbiamo dimostrare due cose:
            \begin{enumerate}
                \item l'insieme di convergenza semplice è un intervallo ``privo di buchi'' centrato in \(x_0\);
                \item nell'intervallo di convergenza, la convergenza è assoluta.
            \end{enumerate}

            Partiamo dal punto primo. Chiamiamo \(E\) l'insieme di convergenza puntuale della serie di potenze data. È sufficiente
            dimostrare l'implicazione
            \begin{equation}
                x \in \mathrm{E} \implies \forall \, y \; | \; |y - x_0| < |x - x_0|, \, y \in \mathrm{E}. \tag{*}\label{eq:pot*}
            \end{equation}
            Data \(y\) che soddisfa l'implicazione, devo dimostrare che
            \[
                \sum_{n = 0}^{+\infty} a_n \, (y - x_0)^n
            \]
            converge.

            Per prima cosa osserviamo che \(x \in \mathrm{E}\) implica
            \begin{equation}
                \lim_{n \to +\infty} |a_n \, (x - x_0)| = 0, \tag{**}\label{eq:pot**}
            \end{equation}
            in quanto, poiché la serie numerica \(\sum_{n = 0}^{+\infty} a_n \, (x - x_0)^n\) converge, il suo termine generale tende a
            zero. Poi calcoliamo
            \[
                \left|a_n \, (y - x_0)^n\right| = \underbrace{\left|a_n \, (x - x_0)^n\right|}_{\leqslant 1 \text{ per } n \text{ grande
                grazie a \hyperref[eq:pot**]{(**)}}} \times \left|\frac{(y - x_0)^n}{(x - x_0)^n}\right|.
            \]
            Quindi,
            \[
                \left|a_n \, (y - x_0)^n\right| \leqslant \left|\frac{y - x_0}{x - x_0}\right|^n
            \]
            per \(n\) grande.

            Grazie a \hyperref[eq:pot*]{(*)}, la serie
            \[
                \sum_{n = 0}^{+\infty} \left|\frac{y - x_0}{x - x_0}\right|^n
            \]
            è convergente, in quanto è una geometrica con ragione \(q < 1\). Quindi, per il teorema del confronto, la serie
            \[
                \sum_{n = 0}^{+\infty} \left|a_n \, (y - x_0)^n\right|
            \]
            è convergente, cioè \(y \in \mathrm{E}\). Abbiamo dunque dimostrato il punto primo. Il punto secondo è stato anch'esso
            dimostrato, in quanto sono stati usati i valori assoluti.
        \end{proof}

        Introduciamo ora il teorema che sancisce come calcolare questo raggio di convergenza.

        \begin{teorema}[Calcolo del raggio di convergenza di una serie di potenze reale]
            Data la serie di potenze 
            \[
                \sum_{x = 0}^{+\infty} a_n \, (x - x_0)^n,
            \]
            possiamo calcolare il raggio di convergenza \(R\) come
            \[
                R = \lim_{n \to +\infty} \left|\frac{a_n}{a_{n+1}}\right|
            \]
            o, alternativamente, come
            \[
                R = \lim_{n \to +\infty} \frac{1}{\sqrt[\leftroot{-1}n]{|a_n|}},
            \]
            dove \(R\) esiste ed appartiene all'insieme \([0, \, +\infty]\).
        \end{teorema}
        \begin{osservazione}
            Il criterio del rapporto ed il criterio della radice per le serie numeriche forniscono \(\frac{1}{R}\).
        \end{osservazione}

        A questo punto, definiamo un paio di serie importanti.

        \begin{definizione}[Serie esponenziale]
            La serie di potenze
            \[
                \sum_{n = 0}^{+\infty} \frac{x^n}{n!} = 1 + x + \frac{x^2}{2} + \frac{x^3}{6} + \frac{x^4}{24} + \ldots,
            \]
            avente \(x_0 = 0\) e \(a_n = \frac{1}{n!}\), è detta \textbf{serie esponenziale}.

            Il suo raggio di convergenza è dato da
            \[
                R = \lim_{n \to +\infty} \frac{a_n}{a_{n+1}} = \lim_{n \to +\infty} \frac{1}{n!} (n + 1)! = \lim_{n \to +\infty} (n + 1) = +\infty.
            \]
            La serie esponenziale converge dunque assolutamente \(\forall \, x \in \R\).
        \end{definizione}

        \begin{definizione}[Serie logaritmica]
            La serie di potenze
            \[
                \sum_{n = 1}^{+\infty} \frac{x^n}{n} = x + \frac{x^2}{2} + \frac{x^3}{3} + \frac{x^4}{4} + \ldots,
            \]
            avente \(x_0 = 0\) e \(a_n = \frac{1}{n}\), con \(n \geqslant 1\), è detta \textbf{serie logaritmica}.

            Il suo raggio di convergenza è dato da
            \[
                R = \lim_{n \to +\infty} \frac{1}{\left|\sqrt[\leftroot{-1}n]{a_n}\right|} = \lim_{n \to +\infty} \frac{1}{\sqrt[\leftroot{-1}n]{\frac{1}{n}}} = \lim_{n \to +\infty} \sqrt[\leftroot{-1}n]{n} = 1.
            \]
            La serie logaritmica converge dunque assolutamente per \(|x| < 1\) e non converge per \(|x| > 1\). Per quanto riguarda
            il caso \(|x| = 1\), se \(x = 1\) la serie è una serie armonica divergente, mentre con \(x = -1\) la serie converge
            per il criterio di Leibniz.

            In conclusione, la serie logaritmica ha insieme di convergenza semplice \([-1, \, 1)\) ed insieme di convergenza
            assoluta \((-1, \, 1)\).
        \end{definizione}

        Terminiamo il capitolo introducendo gli ultimi teoremi.

        \begin{teorema}[Convergenza totale per una serie di potenze reale]
            Data una generica serie di potenze
            \[
                \sum_{n = 0}^{+\infty} a_n \, (x - x_0)^n,
            \]
            avente raggio di convergenza \(0 < R \leqslant +\infty\), si ha:
            \begin{itemize}
                \item Se \(R = +\infty\), la serie converge totalmente in ogni intervallo chiuso \([a, \, b]\),
                    \(\forall \, a, \, b \in \R\), con \(a < b\).
                \item Se \(0 < R < +\infty\), la serie converge totalmente in ogni intervallo chiuso \([a, \, b] \subset (x_0 - R, \, x_0 + R)\).
            \end{itemize}
        \end{teorema}

        \begin{teorema}[Integrabilità e derivabilità termine a termine per una serie di potenze reale]
            Sia data una generica serie di potenze reale
            \[
                \sum_{n = 0}^{+\infty} a_n \, (x - x_0)^n,
            \]
            avente raggio di convergenza \(0 < R \leqslant +\infty\) e somma \(f(x)\), con \(x \in (x_0 - R, \, x_0 + R)\). Allora:
            \begin{enumerate}
                \item Per ogni \(x \in (x_0 - R, \, x_0 + R)\) si ha
                    \[
                        \intDef{x_0}{x}{f(s)}{s} \coloneqq \intDef{x_0}{x}{\sum_{n = 0}^{+\infty} a_n \, (s - x_0)^n}{s} = \sum_{n = 0}^{+\infty} a_n \, \intDef{x_0}{x}{(s - x_0)^n}{s}
                    \]
                    ed il raggio di convergenza della serie integrata è anch'esso \(R\).
                \item \(f(x)\) è derivabile, dunque in particolare anche continua, in \((x_0 - R, \, x_0 + R)\). Inoltre,
                    \[
                        f'(x) \coloneqq \left(\sum_{n = 0}^{+\infty} a_n \, (x - x_0)^n\right)' = \sum_{n = 0}^{+\infty} \left(a_n \, (x - x_0)^n\right)'
                    \]
                    ed il raggio di convergenza della serie derivata è anch'esso \(R\).
            \end{enumerate}
        \end{teorema}
        \begin{osservazione}
            La serie di potenze è in realtà derivabile ad ogni ordine, mantenendo il raggio di convergenza.
        \end{osservazione}
        \begin{osservazione}
            Il comportamento della serie in \(x = x_0 \pm R\) può variare, dunque va studiato separatamente.
        \end{osservazione}

\end{document}