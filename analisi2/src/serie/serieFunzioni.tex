% Imposto la radice del documento, utile per Visual Studio Code ed altri editor
%! TEX root = ../../analisi2.tex

% Imposto il file come sottofile del documento principale
\documentclass[../../analisi2]{subfiles}

\begin{document}

    \chapter{Serie di funzioni}

        Introduciamo il concetto di serie di funzione.

        \begin{definizione}[Serie di funzioni]
            Dato un intervallo \(\J \subseteq \R\), siano, \(\forall \, n \in \mathbb{N}\), \(f_n : \J \to \R\) delle
            funzioni. La serie di funzioni di termine generale \(f_n(x)\) è la successione delle somme parziali:
            \begin{align*}
                S_0 (x) &= f_0 (x),\\
                S_1 (x) &= f_0 (x) + f_1 (x),\\
                &\dots\\
                S_k (x) &= \sum_{n = 0}^k f_n (x).
            \end{align*}
        \end{definizione}
        \begin{osservazione}
            Fissato un punto \(\overline{x} \in \J\), si ha che la serie di termine generale \(f_n (\overline{x})\) è una serie
            numerica. Abbiamo quindi infinite serie numeriche, una per ogni \(x \in \J\).
        \end{osservazione}

        Parliamo dunque della convergenza.
        
        \begin{definizione}[Convergenza puntuale di una serie di funzioni]
            Diciamo che la serie di funzioni di termine generale \(f_n (x)\), \(x \in \J\), \textbf{converge puntualmente} o
            semplicemente nel punto \(\overline{x} \in \J\) se la serie numerica di termine generale \(f_n (\overline{x})\) è
            convergente, ovvero se esiste finito il limite
            \[
                \lim_{k \to +\infty} S_k(\overline{x}) = \lim_{k \to +\infty} \sum_{n = 0}^k f_n (\overline{x}).
            \]
        \end{definizione}
        \begin{osservazione}
            Una serie di funzioni può essere convergente per alcuni \(x \in \J\), divergente od indeterminata per altri
            \(x \in \J\).
        \end{osservazione}

        Da questa ultima osservazione deriva la prossima definizione.
        
        \begin{definizione}[Insieme di convergenza puntuale e somma della serie]
            L'insieme \(\mathrm{E} \subseteq \J\) dei punti \(x\) in cui la serie converge è detto \textbf{insieme di
            convergenza puntuale} della serie. Nell'insieme \(\mathrm{E}\) risulta definita la \textbf{somma della serie}:
            \[
                f(x) = \sum_{n = 0}^{+\infty} f_n(x) = \lim_{k \to +\infty} S_k (x), \quad \text{con } x \in \mathrm{E}.
            \]
        \end{definizione}

        Un altro tipo di convergenza, simile a quella puntuale, è quella assoluta: per ogni fissato \(\overline{x} \in \J\)
        guardiamo la convergenza della serie numerica \(\left|f_n(\overline{x})\right|\).
        
        \begin{definizione}[Convergenza assoluta di una serie di funzioni]
            Diciamo che la serie di funzioni di termine generale \(f_n(x)\), con \(x \in \J\), \textbf{converge assolutamente}
            nel punto \(\overline{x} \in \J\) se la serie numerica di termine generale \(\left|f_n(\overline{x})\right|\) è
            convergente.
        \end{definizione}
        \begin{osservazione}
            La convergenza assoluta implica quella semplice, ovvero, se una serie converge assolutamente in \(\overline{x}\) allora
            \(\overline{x}\) appartiene al suo insieme di convergenza puntuale.
        \end{osservazione}

        Esiste un altro tipo di convergenza, che prende in considerazione un intervallo.
        
        \begin{definizione}[Convergenza totale di una serie]
            La serie di termine generale \(f_n(x)\), con \(x \in \J\), \textbf{converge totalmente} in
            \(\I \subseteq \J\) se:
            \begin{enumerate}
                \item si ha \(\left|f_n(x)\right| \leqslant a_n, \, \forall \, x \in \I, n \in \N\),
                \item la serie numerica di termine generale \(a_n\) è convergente.
            \end{enumerate}
        \end{definizione}
        \begin{osservazione}
            La convergenza totale in \(\I\) implica la convergenza assoluta per ogni \(\overline{x} \in \I\), che a
            sua volta implica quella puntuale. Questa implicazione non si applica però all'inverso.
        \end{osservazione}
        \begin{osservazione}
            Se le \(f_n\) sono continue e la serie converge totalmente, allora la funzione somma è continua ed integrabile termine
            a termine.
        \end{osservazione}

\end{document}