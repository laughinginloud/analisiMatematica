% Imposto la radice del documento, utile per Visual Studio Code ed altri editor
%! TEX root = ../../analisi1.tex

% Imposto il file come sottofile del documento principale
\documentclass[../../analisi1]{subfiles}

\begin{document}

    \chapter{Giustificazione della formula di Eulero con l’esponenziale complesso}
    \label{formulaEulero}

        \section*{Definizioni necessarie}

            \begin{itemize}
                \item Si ricorda che il \textbf{Polinomio di Taylor (\(T _n ^ f (x) \))} è così definito:

                        \[ \sum_{k = 0}^{n} \frac{f^{(k)} (x_0)}{k!}(x-x_0)^k\]
                    
                \item Data la funzione:
                        \begin{align*}
                            f_k : A = [a, b] &\longrightarrow \mathbb{R}\\
                                         x &\longmapsto y = f_k(x) 
                        \end{align*}
    
                        Si dice \textbf{serie di funzioni}:

                        \[  \sum_{k=0}^{+\infty} f_k(x) \]

                        Un esempio di \textbf{serie di funzioni} è il \textbf{Polinomio di Taylor} esteso a \(+\infty\).

                \item La funzione \(f_k(x)\) è detta \emph{argomento} della serie.
                \item La \textbf{successione delle somme parziali} è così definita:
                      
                        \[  S_N (x) = \sum_{n=0}^{N} f_k(x) \]
                    
                \item Il \emph{carattere} (o la \emph{natura}) della \textbf{serie di funzioni} è il \emph{carattere} (o la \emph{natura}) della sua \textbf{successione delle somme parziali}.
                \item Se:
                        \[  \forall \, x^* \in [a, b] \qquad \lim_{N \to +\infty} S_N (x^*) = L(x^*)    \]

                        La serie di funzioni converge puntualmente in tutto \(A = [a, b]\).
            \end{itemize}
           

        \section*{Enunciato}

            Ridefinendo le funzioni \( e^x, \sin x, \cos x \) nei complessi è possibile verificare la \textbf{formula di Eulero}:

                \[  e^{i\vartheta} = \cos \vartheta + i \sin \vartheta \]

        \section*{Dimostrazione}

            \subsection*{Parte 1 - \(e^z\)}    

                Consideriamo la seguente serie di funzioni:
                \[  \sum_{k=\dots}^{+\infty} \frac{1}{k!} \times x^k \qquad x \in \mathbb{R}  \]

                Questo è lo sviluppo di MacLaurin di \(e^x\) esteso all'infinito. Portiamo ora la serie nei complessi:
                \[  \sum_{k=\dots}^{+\infty} \frac{1}{k!} \times z^k \qquad x \in \mathbb{C}  \]

                Verifichiamo se:
                \begin{itemize}
                    \item converge puntualmente in \(z^* \qquad \forall \, z^*  \in \mathbb{C}\);
                    \item converge assolutamente puntualmente in \(z^* \qquad \forall \, z^*  \in \mathbb{C}\).
                \end{itemize}
            
                Sappiamo che se converge assolutamente ne seguirà la convergenza semplice. Quindi passiamo a dimostrare che:

                \[  \sum_{k=\dots}^{+\infty} \left| \frac{1}{k!} \times z^k \right| = \sum_{k=\dots}^{+\infty} A_k \]

                Applico il \textbf{\hyperref[criterioRapportoSerie]{criterio del rapporto}}:
                \begin{align*}
                    \lim_{k \to +\infty} \frac{A_{k+1}}{A_k} &= \lim_{k \to +\infty} \frac{\left|(z^*)^{k+1}\right|}{(k+1)k!} \times \frac{k!}{\left|(z^*)^k\right|} \\
                    &= \lim_{k \to +\infty} \frac{z^*}{k+1} = 0
                \end{align*}

                Quindi \(\sum_{k=\dots}^{+\infty} A_k\) converge puntualmente (\(\forall \, z^* \in \mathbb{C}\)) 
                e la serie  \(\sum_{k=\dots}^{+\infty} \frac{1}{k!} \times z^k \) converge assolutamente e semplicemente puntualmente.

                Questa serie corrisponde quindi a una funzione di variabile complessa \(f(z)\).

                Definiamo così la funzione:
                \[  e^z \, \stackrel{df}{=} \, \sum_{k=0}^{+\infty} \frac{1}{k!} \times z^k    \]

                Notiamo anche che se \(z = x + 0 \times i\) abbiamo:
                \[  e^x = \sum_{k=0}^{+\infty} \frac{1}{k!} \times x^k \]

                Che altro non è che lo sviluppo di MacLaurin di \(e^x\) esteso a \(+\infty\). 
                Abbiamo così definito la funzione esponenziale nei complessi.

            Lo stesso tipo di procedimento può essere fatto per altre funzioni elementari.
            
            \subsection*{Parte 2 - \(\sin z\)}

                Consideriamo la seguente serie di funzioni:
                \[  \sum_{k=\dots}^{+\infty} \frac{(-1)^k}{(2k+1)!} \times x^{2k+1} \qquad x \in \mathbb{R}  \]

                Questo è lo sviluppo di MacLaurin di \(\sin x\) esteso a \(+ \infty\). Portiamo ora la serie nei complessi:
                \[  \sum_{k=\dots}^{+\infty} \frac{(-1)^k}{(2k+1)!} \times z^{2k+1} \qquad x \in \mathbb{C}  \]

                Verifichiamo se:
                \begin{itemize}
                    \item converge puntualmente in \(z^* \qquad \forall \, z^*  \in \mathbb{C}\)
                    \item converge assolutamente puntualmente in \(z^* \qquad \forall \, z^*  \in \mathbb{C}\)
                \end{itemize}
            
                Sappiamo che se converge assolutamente ne seguirà la convergenza semplice. Quindi passiamo a dimostrare che:

                \[  \sum_{k=\dots}^{+\infty} \left| \frac{(-1)^k}{(2k+1)!} \times z^{2k+1} \right| = \sum_{k=\dots}^{+\infty} B_k \]

                Applico il \textbf{\hyperref[criterioRapportoSerie]{criterio del rapporto}}:
                \begin{align*}
                    \lim_{k \to +\infty} \frac{B_{k+1}}{B_k} &= \lim_{k \to +\infty} \frac{\left| (-1)^{k+1} \times (z^*)^{2(k+1)+1}\right|}{\left[2(k+1)+1\right]!} \times \frac{\left(2k+1\right)!}{\left| (-1)^k \times (z^*)^{2k+1}\right|} \\
                    &= \lim_{k \to +\infty} \frac{(z^*)^2}{(2k+3)(2k+2)} = 0
                \end{align*}

                Quindi \(\sum_{k=\dots}^{+\infty} B_k\) converge puntualmente (\(\forall \, z^* \in \mathbb{C}\)) 
                e la serie  \(\sum_{k=\dots}^{+\infty} \frac{(-1)^k}{(2k+1)!} \times z^{2k+1}\) converge assolutamente e semplicemente puntualmente.

                Questa serie corrisponde quindi a una funzione di variabile complessa \(f(z)\).

                Definiamo così la funzione:
                \[  \sin z \stackrel{df}{=} \sum_{k=0}^{+\infty} \frac{(-1)^k}{(2k+1)!} \times z^{2k+1}    \]

                Notiamo anche che se \(z = x + 0 \times i\) abbiamo:
                \[  \sin x = \sum_{k=0}^{+\infty} \frac{(-1)^k}{(2k+1)!} \times x^{2k+1} \]

                Che altro non è che lo sviluppo di Mac Laurin di \(\sin x\) esteso a \(+\infty\). 
                Abbiamo così definito la funzione seno nei complessi.

            \subsection*{Parte 3 - \(\cos z\)}

                Consideriamo la seguente serie di funzioni:
                \[  \sum_{k=\dots}^{+\infty} \frac{(-1)^k}{(2k)!} \times x^{2k} \qquad x \in \mathbb{R}  \]

                Questo è lo sviluppo di MacLaurin di \(\cos x\) esteso a \(+ \infty\). Portiamo ora la serie nei complessi:
                \[  \sum_{k=\dots}^{+\infty} \frac{(-1)^k}{(2k)!} \times z^{2k} \qquad x \in \mathbb{C}  \]

                \newpage

                Verifichiamo se:
                \begin{itemize}
                    \item converge puntualmente in \(z^* \qquad \forall \, z^*  \in \mathbb{C}\)
                    \item converge assolutamente puntualmente in \(z^* \qquad \forall \, z^*  \in \mathbb{C}\)
                \end{itemize}
            
                Sappiamo che se converge assolutamente ne seguirà la convergenza semplice. Quindi passiamo a dimostrare che:

                \[  \sum_{k=\dots}^{+\infty} \left| \frac{(-1)^k}{(2k)!} \times z^{2k} \right| = \sum_{k=\dots}^{+\infty} C_k \]

                Applico \textbf{\hyperref[criterioRapportoSerie]{criterio del rapporto}}:
                \begin{align*}
                    \lim_{k \to +\infty} \frac{C_{k+1}}{C_k} &= \lim_{k \to +\infty} \frac{\left| (-1)^{k+1} \times (z^*)^{2(k+1)}\right|}{\left[2(k+1)\right]!} \times \frac{\left(2k\right)!}{\left| (-1)^k \times (z^*)^{2k}\right|} \\
                    &= \lim_{k \to +\infty} \frac{(z^*)^2}{(2k+2)(2k+1)} = 0
                \end{align*}

                Quindi \(\sum_{k=\dots}^{+\infty} C_k\) converge puntualmente (\(\forall \, z^* \in \mathbb{C}\)) 
                e la serie  \(\sum_{k=\dots}^{+\infty} \frac{(-1)^k}{(2k)!} \times z^{2k}\) converge assolutamente e semplicemente puntualmente.

                Questa serie corrisponde quindi a una funzione di variabile complessa \(f(z)\).

                Definiamo così la funzione:

                \[  \cos z \stackrel{df}{=} \sum_{k=0}^{+\infty} \frac{(-1)^k}{(2k)!} \times z^{2k}    \]

                Notiamo anche che se \(z = x + 0 \times i\) abbiamo:

                \[  \cos x = \sum_{k=0}^{+\infty} \frac{(-1)^k}{(2k)!} \times x^{2k} \]

                Che altro non è che lo sviluppo di MacLaurin di \(\cos x\) esteso a \(+\infty\). 
                Abbiamo così definito la funzione seno nei complessi.

            \subsection*{Parte 4 - \textbf{La formula di Eulero}}

                Abbiamo ora definito le funzioni \(e^z\), \(\sin z\), \(\cos z\) in \(\mathbb{C}\) nel modo seguente:
                \[  e^z  =  \sum_{k=0}^{+\infty} \frac{1}{k!} \times z^k    \]
                \[  \sin z = \sum_{k=0}^{+\infty} \frac{(-1)^k}{(2k+1)!} \times z^{2k+1} \]
                \[  \cos z = \sum_{k=0}^{+\infty} \frac{(-1)^k}{(2k)!} \times z^{2k}    \]

                Prendiamo ora \( z = i\vartheta \) (parte reale nulla) avremo:
                \begin{align*}
                    e^{i\vartheta}  &=  \sum_{k=0}^{+\infty} \frac{1}{k!} \times (i\vartheta)^k \\
                    &= \frac{1}{0!} \times (i\vartheta)^0 + \frac{1}{1!} \times (i\vartheta)^1 + \frac{1}{2!} \times (i\vartheta)^2 +  \frac{1}{3!} \times (i\vartheta)^3 + \dots  \qquad (i^2 = -1)\\
                    &= 1 + i\vartheta - \frac{1}{2}\vartheta^2 - \frac{1}{3!}i\vartheta^3 + \frac{1}{4!}\vartheta^4 + \frac{1}{5!}i\vartheta^5 + \dots
                \end{align*}
                
                La convergenza assouluta autorizza ad usare le proprietà elementari della somma. 
                Commuto quindi tutti i termini con la \(i\) in fondo e gli altri li porto avanti. Ottengo così:
                \[  e^{i\vartheta} =\sum_{k=0}^{+\infty} \frac{(-1)^k}{(2k)!} \times \vartheta^{2k} + i\left(\sum_{k=0}^{+\infty} \frac{(-1)^k}{(2k+1)!} \times \vartheta^{2k+1}  \right)\]

                Da cui:
                \[ e^{i\vartheta} = \cos \vartheta + i \sin \vartheta \]

                c.v.d.
\end{document}