% Imposto la radice del documento, utile per Visual Studio Code ed altri editor
%! TEX root = ../../analisi1.tex

% Imposto il file come sottofile del documento principale
\documentclass[../../analisi1]{subfiles}

\begin{document}

    \chapter{Numerabilità di \(\mathbb{Z}\) e \(\mathbb{Q}\) e non numerabilità di \(\mathbb{R}\)}

        \section*{Numerabilità di \(\mathbb{Z}\)}

            \(\mathbb{Z}\) è numerabile, in quanto è facilmente possibile metterlo in corrispondenza biunivoca con \(\mathbb{N}\).
            
            \begin{table}[h]
                \centering
                \begin{tabular}{>{$}c<{$} >{$}c<{$} >{$}c<{$} >{$}c<{$} >{$}c<{$} >{$}c<{$} >{$}c<{$} >{$}c<{$} >{$}c<{$} >{$}c<{$}}
                    \newline\vspace*{1mm}\newline
                    \mathbb{Z} & 0 & 1 & -1 & 2 & -2 & \dots & n & -n & \dots\\
                    \newline\vspace*{1mm}\newline
                    & \bigg\updownarrow & \bigg\updownarrow & \bigg\updownarrow & \bigg\updownarrow & \bigg\updownarrow & & \bigg\updownarrow & \bigg\updownarrow\\
                    \mathbb{N} & 0 & 1 & 2 & 3 & 4 & \dots & 2n-1 & 2n & \dots
                \end{tabular}
            \end{table}
            
            Dunque, \(\mathbb{Z}\) è numerabile.

        \section*{Numerabilità di \(\mathbb{Q}\)}

            Cominciamo a scrivere l'insieme \(\mathbb{Q}\) come una tabella.

            \begin{table}[h]
                \centering
                \begin{tabular}{>{$}c<{$} >{$}c<{$} >{$}c<{$} >{$}c<{$} >{$}c<{$} >{$}c<{$} >{$}c<{$}}
                    0 & 1 & 2 & 3 & 4 & 5 & \dots\\
                    0 & -1 & -2 & -3 & -4 & -5 & \dots\\
                    %\newline\vspace*{1mm}\newline
                    0 & \sfrac{1}{2} & \sfrac{2}{2} & \sfrac{3}{2} & \sfrac{4}{2} & \sfrac{5}{2} & \dots\\
                    %\newline\vspace*{1mm}\newline
                    0 & -\sfrac{1}{2} & -\sfrac{2}{2} & -\sfrac{3}{2} & -\sfrac{4}{2} & -\sfrac{5}{2} & \dots\\
                    %\newline\vspace*{1mm}\newline
                    0 & \sfrac{1}{3} & \sfrac{2}{3} & \sfrac{3}{3} & \sfrac{4}{3} & \sfrac{5}{3} & \dots\\
                    0 & -\sfrac{1}{3} & -\sfrac{2}{3} & -\sfrac{3}{3} & -\sfrac{4}{3} & -\sfrac{5}{3} & \dots\\
                    \dots & \dots & \dots & \dots & \dots & \dots & \dots
                \end{tabular}
            \end{table}

            Possiamo dunque costruire una successione, prendendo gli elementi sulle diagonali a partire dall'angolo in alto a sinistra,
            evitando i doppioni.

            \begin{table}[h]
                \centering
                \begin{tabular}{>{$}c<{$} >{$}c<{$} >{$}c<{$} >{$}c<{$} >{$}c<{$} >{$}c<{$} >{$}c<{$} >{$}c<{$} >{$}c<{$} >{$}c<{$}}
                    \newline\vspace*{1mm}\newline
                    \mathbb{Q} & 0 & 1 & 2 & -1 & 3 & -2 & \sfrac{1}{2} & \dots\\
                    \newline\vspace*{1mm}\newline
                    & \bigg\updownarrow & \bigg\updownarrow & \bigg\updownarrow & \bigg\updownarrow & \bigg\updownarrow & \bigg\updownarrow & \bigg\updownarrow\\
                    \mathbb{N} & 0 & 1 & 2 & 3 & 4 & 5 & 6 & \dots
                \end{tabular}
            \end{table}

            Dunque, anche \(\mathbb{Q}\) è numerabile.

        \newpage

        \section*{Non numerabilità di \(\mathbb{R}\)}

            Cominciamo scrivendo una sequenza infinita di numeri razionali \(\in [0, 1]\).
            \begin{align*}
                0&.1234\dots\\
                0&.9876\dots\\
                0&.1928\dots\\
                &\dots
            \end{align*}

            Se noi definiamo un nuovo numero, prendendo una cifra alla volta in diagonale ed incrementandola di uno,
            ad esempio \(0.293\dots\), questo sarà, per definizione, diverso da ogni altro numero della sequenza.
            Non è pertanto possibile mettere in corrispondenza biunivoca \(\mathbb{R}\) con \(\mathbb{N}\), dunque \(\mathbb{R}\)
            non è numerabile.

\end{document}