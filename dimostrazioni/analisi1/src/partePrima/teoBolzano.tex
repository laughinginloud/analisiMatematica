% Imposto la radice del documento, utile per Visual Studio Code ed altri editor
%! TEX root = ../../analisi1.tex

% Imposto il file come sottofile del documento principale
\documentclass[../../dimostrazioni]{subfiles}

\begin{document}

    \chapter{Teorema di Bolzano}
    \label{teoBolzano}

        \section*{Enunciato}

            \subsection*{Ipotesi}

                Sia \(f(x)\) una funzione tale che
                \begin{align*}
                    f : A \subseteq \mathbb{R} &\longrightarrow \mathbb{R}\\
                    x &\longmapsto y = f(x) 
                \end{align*}

                Supponiamo inoltre che:

                \begin{enumerate}
                    \indentitem \item \(A = [a, b]\) è un intervallo compatto;
                    \indentitem \item \(f\) è continua su \(A\);
                    \indentitem \item \(f(a) f(b) < 0\).
                \end{enumerate}

            \subsection*{Tesi}
            
                \[\exists \, x^* \in (a,b) \; | \; f(x^*) = 0 \]

        \section*{Dimostrazione}

            Per fissare le idee,
            \begin{align*}
                f(a) &> 0\\
                f(b) &< 0
            \end{align*}

            Procediamo per bisezione: definiamo \(x_1\) uguale al punto medio, ovvero
            \[
                x_1 = \frac{a_0 + b_0}{2}
            \]

            A questo punto, valutiamo \(f(x_1)\):
            \[
                f(x_1) =
                \left\{
                \begin{aligned}
                    &\text{se } f(x_1) = 0 && \text{\emph{ho trovato lo zero}}\\
                    &\text{se } f(x_1) > 0 && a_1 = x_1 \land b_1 = b_0 \quad \text{\emph{(studio l'intervallo destro)}}\\
                    &\text{se } f(x_1) < 0 && a_1 = a_0 \land b_1 = x_1 \quad \text{\emph{(studio l'intervallo sinistro)}}
                \end{aligned}
                \right.
            \]

            Possiamo quindi proseguire fino a \(x_k\), che sarà definito come
            \[
                x_k = \frac{a_{k - 1} + b_{k - 1}}{2}
            \]
            da cui
            \[
                f(x_k) =
                \left\{
                \begin{aligned}
                    &\text{se } f(x_k) = 0 && \text{\emph{ho trovato lo zero}}\\
                    &\text{se } f(x_k) > 0 && a_k = x_k \land b_k = b_{k - 1} \quad \text{\emph{(studio l'intervallo destro)}}\\
                    &\text{se } f(x_k) < 0 && a_k = a_{k - 1} \land b_k = x_k \quad \text{\emph{(studio l'intervallo sinistro)}}
                \end{aligned}
                \right.
            \]

            Abbiamo dunque due succesioni:
            \begin{gather*}
                a_k \text{\emph{ monotona crescente}}\\
                b_k \text{\emph{ monotona decrescente}}
            \end{gather*}
            entrambe limitate, in quanto stanno in \([a, b]\), e convergenti, per il \hyperref[teoFondSuccMono]{teorema fondamentale
            delle successioni monotone}.
            Quindi,
            \[
                \left.
                \begin{aligned}
                    a_k &\to L\\
                    b_k &\to M
                \end{aligned}
                \right.
                \quad \text{\emph{ma }} L = M \text{\emph{ dato che }} \mathrm{dist}(a_k, b_k) = \frac{b - a}{2^k} \xrightarrow[k \to +\infty]{} 0
            \]

            Prendiamo quindi \(x^* = L = M\) e mostriamo che \(f(x^*) = 0\):
            \[
                \left.
                \begin{aligned}
                    f(a_k) \text{\emph{ converge ad un valore }} \geqslant 0\\
                    f(b_k) \text{\emph{ converge ad un valore }} \leqslant 0
                \end{aligned}
                \right\}
                \text{\emph{ per il \hyperref[teoPermSegno]{teorema della permanenza del segno}}}
            \]

            Uso la continuità di \(f\): se \(x \to x_0\), allora \(f(x) \to f(x_0)\). Dunque,
            \[
                \left.
                \begin{aligned}
                    f(a_k) \to f(x^*) \geqslant 0&\\
                    f(b_k) \to f(x^*) \leqslant 0&
                \end{aligned}
                \right\} \, f(x^*) = 0
            \]
        
\end{document}