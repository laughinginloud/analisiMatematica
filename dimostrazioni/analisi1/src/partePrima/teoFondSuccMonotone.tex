% Imposto la radice del documento, utile per Visual Studio Code ed altri editor
%! TEX root = ../../analisi1.tex

% Imposto il file come sottofile del documento principale
\documentclass[../../dimostrazioni]{subfiles}

\begin{document}

    \chapter{Teorema fondamentale delle successioni monotone}
    \label{teoFondSuccMono}

        Si dimostrano diversi casi:
        \begin{enumerate}
            \indentitem \item una successione monotona e limitata \emph{converge};
            \indentitem \item una successione monotona e illimitata
                \begin{itemize}
                    \indentitem \item \emph{diverge positivamente}, se crescente;
                    \indentitem \item \emph{diverge negativamente}, se decrescente.
                \end{itemize}
        \end{enumerate}            


        \section*{Primo caso}
            
            \subsection*{Enunciato}
                
                \subsubsection*{Ipotesi}
                         
                Fisso per comodità \(a_n\) monotona crescente

                \begin{enumerate}
                    \indentitem \item \(a_n \leqslant a_{n+1}\) (ipotesi di monotonia);
                    \indentitem \item \(\{a_n\} \subset \mathrm{B}_r(0)\) con \(r > 0 \) (ipotesi di limitatezza).
                \end{enumerate}
                
                \subsubsection*{Tesi}
                    \[\lim_{n \to +\infty} a_n = L\]
            
            \subsection*{Dimostrazione}
                Essendo \(a_n\) limitata avrà un maggiorante in \(r\) e quindi avrà il \(\sup\).
                Dimostro che \(L = \sup{a_n}\).

                Definiamo un intorno \(\varepsilon\) di \(\sup\) come \(\mathrm{B}_\varepsilon (S)\): allora
                \(\forall \varepsilon \geqslant 0, \, S-\varepsilon\) non è più né sup né maggiorante, quindi
                \begin{gather*}
                    \exists \, a_{n^*} \mid S-\varepsilon < a_{n^*} \leqslant S \\
                    \forall n > n^*, \, S-\varepsilon < a_{n^*} \leqslant a_n \leqslant S \\
                    a_n \in \mathrm{B}_\varepsilon(S) 
                \end{gather*}
                
        \newpage

        \section*{Secondo caso}

            \subsection*{Enunciato}

            Consideriamo il caso della crescenza.
                
            \subsubsection*{Ipotesi}
                 
            Fisso per comodità \(a_n\) monotona crescente

            \begin{enumerate}
               \indentitem \item \(a_n \leqslant a_{n+1}\) (ipotesi di monotonia);
               \indentitem \item \(\nexists \, \mathrm{B}_r(0) \supset \{a_n\} \quad \forall n \) (ipotesi di illimitatezza).
            \end{enumerate}
        
            \subsubsection*{Tesi}
                \[\lim_{n \to +\infty} a_n = +\infty\]
    
        \subsection*{Dimostrazione}
            Essendo \(a_n\) illimitata 
            \[\forall \, \mathrm{B}_r(0), \, a_{n^*} \geqslant r \]
            e per la monotonia
            \[\forall n > n^*, \, a_{n^*} \leqslant a_n\]
            quindi, definitivamente
            \[a_n \in \mathrm{B}_r(+\infty)\]
            e per la definizione di limite 
            \[\lim_{n \to +\infty} a_n = +\infty\]

\end{document}