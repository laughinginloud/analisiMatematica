% Imposto la radice del documento, utile per Visual Studio Code ed altri editor
%! TEX root = ../../analisi1.tex

% Imposto il file come sottofile del documento principale
\documentclass[../../analisi1]{subfiles}

\begin{document}

    \chapter{Teorema di unicità del limite}

        \section*{Enunciato}

            Se una successione converge, il valore a cui converge è unico.

        \section*{Dimostrazione}

            Sia \(\{a_n\}\) una successione convergente ed ipotizziamo, per assurdo, che
            \begin{align*}
                \lim_{n \to +\infty} a_n = L_1\\
                \lim_{n \to +\infty} a_n = L_2
            \end{align*}
            con \(L_1 \neq L_2\).

            Per la definizione di limite otteniamo
            \begin{align*}
                &\forall \, \mathrm{B}_r (L_1), \, \exists \, M_1 \; | \; \forall n > M_1, \, a_n \in \mathrm{B}_r (L_1)\\
                &\forall \, \mathrm{B}_r (L_2), \, \exists \, M_2 \; | \; \forall n > M_2, \, a_n \in \mathrm{B}_r (L_2)
            \end{align*}

            Scegliamo \(r\) come
            \begin{align*}
                r &< \frac{\mathrm{dist} (L_1, L_2)}{2}\\
                &< \frac{|L_1 - L_2|}{2}
            \end{align*}
            così da avere
            \[
                \mathrm{B}_r (L_1) \cap \mathrm{B}_r (L_2) = \emptyset
            \]

            Da ciò otteniamo
            \[
                \forall n > \max\{M_1, M_2\}, \, n \in \mathrm{B}_r (L_1) \land n \in \mathrm{B}_r (L_2)
            \]
            il che è assurdo, poiché l'intersezione è equivalente all'insieme vuoto.
    
\end{document}