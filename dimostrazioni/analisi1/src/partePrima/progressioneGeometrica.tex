\documentclass[../../analisi1]{subfiles}

\begin{document}

    \chapter{Progressione geometrica}

        \section*{Enunciato}       

            \[F(n): \sum_{k=0}^{n}q^k \, = \, \frac{1-q^{n+1}}{1-q} \qquad \qquad q \neq 1 \]

        \section*{Dimostrazione}

            Per dimostrare l'enunciato, procediamo con una dimostrazione per induzione.

            \medskip

            % Nota: sono attaccate per evitare multiple righe vuote
            Dimostriamo l'enunciato per \(n = 1\):
            
            \[\sum_{k=0}^{1}q^k \, = \, q^0 + q^1 \, = \, 1 + q\]
            
            Allo stesso modo

            \[\frac{1-q^{1+1}}{1-q} \, = \, \frac{1-q^2}{1-q} \, = \, \frac{(1-q)(1+q)}{1-q} \, = \, 1+q\]

            Che corrisponde al risultato della sommatoria

            Possiamo perciò considerare l'enunciato vero al passo \(n\).

            \medskip

            % Nota: si veda sopra
            Dimostriamolo per \(n + 1\):
            \begin{align*}
                \sum_{k=0}^{n+1}q^k \,=& \, \sum_{k=0}^{n}q^k + q^{n+1} \\
                                =& \, \frac{1-q^{n+1}}{1-q} + q^{n+1} & \text{\emph{Per ipotesi induttiva}} \\
                                =& \, \frac{1-q^{n+1} + q^{n+1} - q^{n+1+1}}{1-q} \\
                                =& \, \frac{1-q^{n+2}}{1-q} \\
            \end{align*}

            Allo stesso modo

            \[\frac{1-q^{(n+1)+1}}{1-q} \, = \, \frac{1-q^{n+2}}{1-q} \]

            Abbiamo quindi dimostrato la progressione geometrica.
    
\end{document}