% Imposto la radice del documento, utile per Visual Studio Code ed altri editor
%! TEX root = ../../analisi1.tex

% Imposto il file come sottofile del documento principale
\documentclass[../../analisi1]{subfiles}

\begin{document}

    \chapter{Cardinalità dell'insieme delle parti}

        \section*{Enunciato}

            Dato un insieme \(\insieme{X}\), l'insieme delle parti \(\insiemeparti{X}\) ha cardinalità pari a \(2^n\), dove \(n\) è il numero di elementi dell'insieme.

        \section*{Dimostrazione}

            Prendiamo un generico insieme \(\insieme{X}\). Questo ha sicuramente come sottoinsiemi \(\emptyset\) e se stesso.\\
            Chiediamoci ora: quanti sottoinsiemi di un elemento possiede? \(\mathrm{C}_{n, 1} = \binom{n}{1}\)\\
            Quanti sottoinsiemi di due elementi possiede? \(\mathrm{C}_{n, 2} = \binom{n}{2}\)\\
            \dots\\
            Quanti sottoinsiemi di \(n - 1\) elementi possiede? \(\mathrm{C}_{n, n - 1} = \binom{n}{n - 1}\)

            A questo punto possiamo dire che
            \[
                \card{\insiemeparti{X}} = 1 + \binom{n}{1} + \binom{n}{2} + \dots + \binom{n}{n - 1} + 1
            \]
            che corrisponde a
            \[
                \card{\insiemeparti{X}} = \sum_{k = 0}^n \binom{n}{k} = {(1 + 1)}^n = 2^n
            \]
            secondo la regola del binomio di Newton. c.v.d.
    
\end{document}