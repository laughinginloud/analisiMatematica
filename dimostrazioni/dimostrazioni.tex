% Importo la struttura del documento
% Imposto la radice del documento, utile per Visual Studio Code ed altri editor
% !TEX root = ../analisi2.tex

% Costanti
\newcommand{\autori}{Mattia Martelli}
\newcommand{\titolo}{Analisi Matematica II}

% Utilizzo il tipo libro impostato per evitare pagine vuote, utili solo per la stampa
\documentclass[oneside]{book}

% Imposto lo stile della numerazione della pagina
\pagestyle{plain}

% Utilizzo il pacchetto per le traduzioni ed imposto la lingua
\usepackage{polyglossia}
\setmainlanguage{italian}

% Imposto il margine delle pagine
\usepackage[margin=1in]{geometry}

% Importo i pacchetti per le immagini con descrizioni, che torneranno utili per le dimostrazioni contenenti disegni
\usepackage{graphicx,subcaption}

% Importo i pacchetti per disegnare
\usepackage{tikz,pgfplots}

% Importo i pacchetti per la matematica
\usepackage{mathtools,amssymb,amsthm,thmtools,cancel}

% Imposto lo stile per l'indice dei teoremi e delle definizioni
\renewcommand{\listtheoremname}{Indice delle Definizioni e dei Teoremi}

% Imposto i collegamenti
\usepackage{hyperref}
\hypersetup{
    unicode=true,
    bookmarksnumbered=true,
    bookmarksopen=false,
    hidelinks,
    pdftitle={\titolo},
    pdfauthor={\autori}
}

% Importo il pacchetto per lo stile degli elenchi
\usepackage{enumitem}

% Importo il pacchetto per lo stile delle tabelle
\usepackage{array}

% Importo il pacchetto per le frazioni in diagonale
\usepackage{xfrac}

% Importo il pacchetto per la personalizzazione del titolo e lo modifico
\usepackage{titlesec}
%\titleformat{\chapter}[display]{\normalfont\LARGE\bfseries}{Dimostrazione numero \thechapter}{.5em}{\Huge}
\renewcommand{\thechapter}{\Roman{chapter}}
\titleformat{\chapter}[display]
    {\bfseries\Large}
    {\filleft\MakeUppercase{\chaptertitlename} \Huge\thechapter}
    {4ex}
    {\titlerule
     \vspace{2ex}%
     \filright}
    [\vspace{2ex}%
     \titlerule]

% Importo il pacchetto per i sottofile
\usepackage{subfiles}

% Importo il paccheto per racchiudere il testo in riquadri
\usepackage{framed}

% Imposto le informazioni presenti sulla copertina
% Nota: date rimuove la propria spaziatura, dato che non deve essere presente
\title{\titolo}
\author{\autori}
\date{\vspace{-5ex}}

\begin{document}

    % Creo la copertina
    \maketitle

    % Creo l'indice
    \tableofcontents

    % Importo il capitolo sulla disuguaglianza di Bernoulli
    \subfile{src/disuguaglianzaBernoulli}

    % Importo il capitolo sul teorema di Fermat
    \subfile{src/teoremaFermat}

    % Importo il capitolo sul teorema di Rolle
    \subfile{src/teoremaRolle}

    % Importo il capitolo sul teorema di Lagrange
    \subfile{src/teoremaLagrange}

    % Importo il capitolo sul test di monotonia
    \subfile{src/testMonotonia}

    % Importo il capitolo sulla cardinalità di R^2
    \subfile{src/cardinalitàRn}

    % Importo il capitolo sul teo di Cauchy
    \subfile{src/teoremaCauchy}

    % Importo il capitolo sul teo de l'Hopital
    \subfile{src/teoremaHopital}

    % Importo il capitolo sul teo resto secondo Peano
    \subfile{src/teorestoPeano}

    % Importo il capitolo sul teo resto secondo Lagrange
    \subfile{src/teorestoLagrange}
    
    % Importo il capitolo sul primo teo fondamentale calcolo integrale 
    \subfile{src/teoFondCalcoloIntegrale1}

    % Importo il capitolo sul teo del valor medio integrale 
    \subfile{src/teoValorMedioIntegrale}

    % Importo il capitolo sul secondo teo fondamentale calcolo integrale 
    \subfile{src/teoFondCalcoloIntegrale2}

    % Importo il capitolo sulla condizione necessaria per la convergenza di una serie
    \subfile{src/CNconvserie.tex}

    % Importo il capitolo sul criterio del rapporto
    \subfile{src/criterioRapportoSerie.tex}

    % Importo il capitolo sul criterio del confronto
    \subfile{src/serieconfronto.tex}

    % Importo il capitolo sulla giustificazione della formula di Eulero con l’esponenziale complesso
    \subfile{src/formulaEulero.tex}
    
\end{document}