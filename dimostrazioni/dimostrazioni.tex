% Importo le costanti
%% Imposto la radice del documento, utile per Visual Studio Code ed altri editor
% !TEX root = ../analisi2.tex

% Comando per l'indentazione degli elementi di un elenco
\newcommand{\indentitem}{\setlength\itemindent{25pt}}

% Comandi per le notazioni insiemistiche
\newcommand{\insieme}[1]{\overline{\rm \underline{#1}}}
\newcommand{\insiemeparti}[1]{\mathcal{P}\left(\insieme{#1}\right)}
\newcommand{\card}[1]{\left| #1 \right|}

% Evito che PGF vada in modalità legacy
\pgfplotsset{compat=1.16}

% Definisco le punte stondate in entrambi i versi [(-) e )-(]
\pgfarrowsdeclare{(}{)}{}
{
  \pgfsetroundcap{}
  \pgfpathmoveto{\pgfpoint{-1.5pt}{-1.5pt}}
  \pgfpatharc{90}{270}{-1.5pt}
  \pgfusepathqstroke{}
}
\pgfarrowsdeclare{)}{(}{}
{
  \pgfsetroundcap{}
  \pgfpathmoveto{\pgfpoint{1.5pt}{1.5pt}}
  \pgfpatharc{90}{270}{1.5pt}
  \pgfusepathqstroke{}
}

% Importo la struttura del documento
% Imposto la radice del documento, utile per Visual Studio Code ed altri editor
% !TEX root = ../dimostrazioni.tex

\newcommand{\autori}{Virginia Longo, Giovanni Manfredi e Mattia Martelli}
\newcommand{\titolo}{Dimostrazioni di Analisi matematica 1}

% Utilizzo il tipo libro impostato per evitare pagine vuote, utili solo per la stampa
\documentclass[oneside]{book}

% Imposto lo stile della numerazione della pagina
\pagestyle{plain}

% Utilizzo il pacchetto per le traduzioni ed imposto la lingua
\usepackage{polyglossia}
\setmainlanguage{italian}

% Imposto il margine delle pagine
\usepackage[margin=1in]{geometry}

% Importo i pacchetti per le immagini con descrizioni, che torneranno utili per le dimostrazioni contenenti disegni
\usepackage{graphicx,subcaption}

% Importo i pacchetti per disegnare
\usepackage{tikz,pgfplots}

% Importo i pacchetti per la matematica
\usepackage{mathtools,amssymb}

% Imposto i collegamenti
\usepackage{hyperref}
\hypersetup{
    unicode=true,
    bookmarksnumbered=true,
    bookmarksopen=false,
    hidelinks,
    pdftitle={\titolo},
    pdfauthor={\autori}
}

% Importo il pacchetto per la personalizzazione del titolo e lo modifico
\usepackage{titlesec}
\titleformat{\chapter}[display]{\normalfont\LARGE\bfseries}{Dimostrazione numero \thechapter}{.5em}{\Huge}

% Importo il pacchetto per i sottofile
\usepackage{subfiles}

% Imposto le informazioni presenti sulla copertina
% Nota: date rimuove la propria spaziatura, dato che non deve essere presente
\title{\titolo}
\author{\autori}
\date{\vspace{-5ex}}

% Importo i comandi personalizzati
% Imposto la radice del documento, utile per Visual Studio Code ed altri editor
% !TEX root = ../analisi2.tex

% Comando per l'indentazione degli elementi di un elenco
\newcommand{\indentitem}{\setlength\itemindent{25pt}}

% Comandi per le notazioni insiemistiche
\newcommand{\insieme}[1]{\overline{\rm \underline{#1}}}
\newcommand{\insiemeparti}[1]{\mathcal{P}\left(\insieme{#1}\right)}
\newcommand{\card}[1]{\left| #1 \right|}

% Evito che PGF vada in modalità legacy
\pgfplotsset{compat=1.16}

% Definisco le punte stondate in entrambi i versi [(-) e )-(]
\pgfarrowsdeclare{(}{)}{}
{
  \pgfsetroundcap{}
  \pgfpathmoveto{\pgfpoint{-1.5pt}{-1.5pt}}
  \pgfpatharc{90}{270}{-1.5pt}
  \pgfusepathqstroke{}
}
\pgfarrowsdeclare{)}{(}{}
{
  \pgfsetroundcap{}
  \pgfpathmoveto{\pgfpoint{1.5pt}{1.5pt}}
  \pgfpatharc{90}{270}{1.5pt}
  \pgfusepathqstroke{}
}

\begin{document}

    % Creo la copertina
    \maketitle

    % Creo l'indice
    \tableofcontents

    % Imposto lo stile interno delle ancore
    \renewcommand*{\theHchapter}{ch.\the\value{chapter}}

    \part{}

        \subfile{src/partePrima/formuleDeMorgan}

        \subfile{src/partePrima/disuguaglianzaBernoulli}

        \subfile{src/partePrima/binomioNewton}

    \part{}

        \subfile{src/parteSeconda/teoremaFermat}

        \subfile{src/parteSeconda/teoremaRolle}

        \subfile{src/parteSeconda/teoremaLagrange}

        \subfile{src/parteSeconda/testMonotonia}

        \subfile{src/parteSeconda/teoremaCauchy}

        \subfile{src/parteSeconda/teoremaHopital}

        \subfile{src/parteSeconda/teorestoPeano}

        \subfile{src/parteSeconda/teorestoLagrange}

        \subfile{src/parteSeconda/teoFondCalcoloIntegrale1}

        \subfile{src/parteSeconda/teoValorMedioIntegrale}

        \subfile{src/parteSeconda/teoFondCalcoloIntegrale2}

        \subfile{src/parteSeconda/CNconvserie.tex}

        \subfile{src/parteSeconda/criterioRapportoSerie}

        \subfile{src/parteSeconda/serieconfronto}

        \subfile{src/parteSeconda/criterioRadiceSerie}

        \subfile{src/parteSeconda/formulaEulero}

    \part*{Dimostrazioni aggiuntive}
    \addcontentsline{toc}{part}{Dimostrazioni aggiuntive}

        % Reinizializzo il contatore del capitolo, cambio lo stile della numerazione, modifico lo stile interno delle ancore
        \setcounter{chapter}{0}
        \renewcommand\thechapter{A\arabic{chapter}}
        \titleformat{\chapter}[display]{\normalfont\LARGE\bfseries}{Dimostrazione aggiuntiva numero \arabic{chapter}}{.5em}{\Huge}
        \renewcommand*{\theHchapter}{chA.\the\value{chapter}}

        \subfile{src/parteExtra/cardinalitàRn}
    
\end{document}