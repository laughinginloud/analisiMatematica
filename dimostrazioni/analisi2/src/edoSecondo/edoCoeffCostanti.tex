% Imposto la radice del documento, utile per Visual Studio Code ed altri editor
%! TEX root = ../../analisi2.tex

% Imposto il file come sottofile del documento principale
\documentclass[../../analisi2]{subfiles}

\begin{document}

    \chapter{Equazioni differenziali omogenee del secondo ordine a coefficienti costanti}

        Per studiare un'equazione differenziale omogenea a coefficienti costanti, possiamo sfruttare il teorema di struttura.
        Partiamo innanzitutto definendo il \textbf{polinomio caratteristico}: data una generica equazione differenziale
        \(a \, y'' + b \, y' + c \, y = 0\), il polinomio caratteristico associato è
        \[
            P(\lambda) = a \, \lambda^2 + b \, \lambda + c.
        \]
        Possiamo dunque facilmente definire l'equazione caratteristica come \(P(\lambda) = 0\), o, più esplicitamente,
        \[
            a \, \lambda^2 + b \, \lambda + c = 0.
        \]
        Abbiamo dunque ricondotto la ricerca delle soluzioni dell'equazione differenziale omogenea a quella delle radici del
        polinomio caratteristico, ovvero alla risoluzione dell'equazione caratteristica. La natura delle radici dipende chiaramente
        dal discriminante \(\Delta = b^2 - 4ac\).

        \section{Il caso \texorpdfstring{\(\Delta > 0\)}{discriminante maggiore di zero}}

            Nel caso di un discriminante positivo, possiamo ricavare due radici \textbf{reali e distinte} tramite la classica formula
            \[
                \lambda_i = \frac{-b \pm \sqrt{\Delta}}{2a},
            \]
            da cui ricaviamo le due soluzioni dell'equazione differenziale come
            \begin{align*}
                y_1(t) &= e^{\lambda_1 \, t},\\
                y_2(t) &= e^{\lambda_2 \, t},
            \end{align*}
            linearmente indipendenti poiché \(\lambda_1 \neq \lambda_2\). Possiamo dunque sfruttare il teorema di struttura ed ottenere
            l'integrale generale dell'equazione di partenza come
            \[
                y(t) = \mathrm{C_1} \, e^{\lambda_1 \, t} + \mathrm{C_2} \, e^{\lambda_2 \, t}.
            \]

        \newpage

        \section{Il caso \texorpdfstring{\(\Delta < 0\)}{discriminante minore di zero}}

            Nel caso di un discriminante negativo, possiamo ricavare due radici \textbf{complesse e coniugate} definite come
            \begin{align*}
                \lambda_1 &= \alpha + i \, \beta,\\
                \lambda_2 &= \alpha - i \, \beta,
            \end{align*}
            dove
            \begin{align*}
                \alpha &= \frac{-b}{2a},\\
                \beta &= \frac{\sqrt{-\Delta}}{2a},
            \end{align*}
            ricavabili dalla formula classica. Prendendo la generica soluzione \(y(t) = e^{\lambda \, t}\) e ricordando la formula
            di Eulero per l'esponenziale complesso \(e^{\alpha + i \, \beta} = e^{\alpha} (\cos \beta + i \, \sin \beta)\), possiamo
            dunque scrivere le soluzioni come
            \begin{align*}
                y_1(t) &= e^{\alpha} (\cos \beta + i \, \sin \beta),\\
                y_2(t) &= e^{\alpha} (\cos \beta + i \, \sin \beta).
            \end{align*}
            Ma poiché cerchiamo soluzioni reali, dobbiamo definire due funzioni che chiameremo \(u\) come
            \begin{align*}
                u_1(t) &= \frac{y_1(t) + y_2(t)}{2},\\
                u_2(t) &= \frac{y_1(t) - y_2(t)}{2},
            \end{align*}
            che possono dunque essere generalizzate come
            \begin{align*}
                u_1(t) &= e^{\alpha \, t} \cos(\beta \, t),\\
                u_2(t) &= e^{\alpha \, t} \sin(\beta \, t).
            \end{align*}
            L'integrale generale è dunque
            \[
                y(t) = e^{\alpha \, t} \left(\mathrm{C_1} \, \cos(\beta \, t) + \mathrm{C_2} \, \sin(\beta \, t)\right).
            \]

        \section{Il caso \texorpdfstring{\(\Delta = 0\)}{discriminante uguale a zero}}

            Nel caso di un discriminante nullo, possiamo ricavare due radici \textbf{reali e concidenti} definite come
            \[
                \lambda_1 = \lambda_2 = -\frac{b}{2a},
            \]
            come si può facilmente ricavare dalla formula classica. Poiché abbiamo una sola soluzione, abbiamo bisogno di cercare una
            funzione \(C(t)\) tale che
            \begin{align*}
                y_2(t) &= C(t) \, e^{\lambda_1},\\
                y_2(t)' &= e^{\lambda_1} \left(C'(t) + \lambda_1 \, C(t)\right),\\
                y_2(t)'' &= e^{\lambda_1} \left(C''(t) + 2\lambda_1 \, C'(t) + 3\lambda_1 \, C(t)\right).
            \end{align*}
            Se sostituiamo nell'equazione differenziale ci accorgiamo che tutti i termini contenenti \(C(t)\) e \(C'(t)\) si
            semplificano, risultando nell'equazione
            \[
                C''(t) = 0 \quad \forall \, t \in \R.
            \]
            La più semplice funzione che soddisfa l'equazione è la funzione identità, ovvero \(C(t) = t\). Possiamo dunque generalizzare
            quanto trovato definendo l'integrale generale come
            \[
                y(t) = \mathrm{C_1} \, e^{\lambda_1 \, t} + \mathrm{C_2} \, t \, e^{\lambda_2 \, t}.
            \]

        \section{Tabella riassuntiva}

            Riassumiamo quanto trovato in una tabella.

            \begin{table}[h!]

                \centering

                \begin{tabular}{||c|c|c|c|c||}
                    \hline
                    \multirow{2}{*}{\(\Delta\)} & \multirow{2}{*}{Radici} & \multicolumn{2}{c|}{Soluzioni} & Integrale generale\\ \cline{3-5}
                    & & \(y_1(t)\) & \(y_2(t)\) & \(\forall \, \mathrm{C_1}, \mathrm{C_2} \in \R\)\\
                    \hline\hline
                    \(\Delta > 0\) & \(\lambda_1 \neq \lambda_2, \; \lambda_1, \, \lambda_2 \in \R\) & \(e^{\lambda_1 \, t}\) & \(e^{\lambda_2 \, t}\) & \(y(t) = \mathrm{C_1} \, e^{\lambda_1 \, t} + \mathrm{C_2} \, e^{\lambda_2 \, t}\)\\
                    \hline
                    \(\Delta = 0\) & \(\lambda_1 = \lambda_2, \; \lambda_1, \, \lambda_2 \in \R\) & \(e^{\lambda_1 \, t}\) & \(t \, e^{\lambda_2 \, t}\) & \(y(t) = \mathrm{C_1} \, e^{\lambda_1 \, t} + \mathrm{C_2} \, t \, e^{\lambda_2 \, t}\)\\
                    \hline
                    \(\Delta < 0\) & \(\lambda_{1, \, 2} = \alpha \pm \beta, \; \alpha, \, \beta \in \R\) & \(e^{\alpha \, t} \cos(\beta \, t)\) & \(e^{\alpha \, t} \sin(\beta \, t)\) & \(y(t) = e^{\alpha \, t} \left(\mathrm{C_1} \, \cos(\beta \, t) + \mathrm{C_2} \, \sin(\beta \, t)\right)\)\\
                    \hline
                \end{tabular}
                
            \end{table}

            Queste formule generali valgono per la risoluzione di equazioni \textbf{omogenee}, siano esse associate o meno, con
            \textbf{coefficienti costanti}. Non si applicano se queste due condizioni non sono verificate.
            
\end{document}