% Imposto la radice del documento, utile per Visual Studio Code ed altri editor
%! TEX root = ../../analisi2.tex

% Imposto il file come sottofile del documento principale
\documentclass[../../analisi2]{subfiles}

\begin{document}

    \chapter{Introduzione alle equazioni differenziali lineari del secondo ordine}

        Un'equazione differenziale del secondo ordine si presenta nella forma
        \[
            a(t) \, y'' + b(t) \, y' + c(t) \, y = f(t),
        \]
        con \(t \in \mathrm{I}\). Definiamo dunque una sua soluzione.

        \begin{definizione}[Soluzione di un'equazione differenziale del secondo ordine]
            Si dice \textbf{soluzione dell'equazione differenziale} nell'intervallo \(\mathrm{I} \subset \R\) una funzione
            \(y : \mathrm{I} \to \R\) \textbf{derivabile due volte} per cui, sostituendo nell'equazione differenziale i valori effettivi
            di \(y(t)\), \(y'(t)\) e \(y''(t)\), si ottiene che
            \[
                a(t) \, y'' + b(t) \, y' + c(t) \, y = f(t) \quad \forall \, t \in \mathrm{I},
            \] cioè un'identità su \(\mathrm{I}\).
        \end{definizione}

        Un'equazione differenziale del secondo ordine ha soluzioni infinite. Queste vengono racchiuse nella loro totalità
        in dipendenza da due parametri all'interno dell'integrale generale. Se a questo aggiungiamo una coppia di condizioni iniziali
        otteniamo una soluzione specifica. Il sistema formato dall'integrale generale e le condizioni iniziali è detto
        \textbf{problema di Cauchy} ed il teorema che garantisce l'unicità della soluzione è detto \textbf{teorema di Cauchy}.

        \begin{teorema}[Teorema di Cauchy]
            Data l'equazione differenziale
            \[
                a(t) \, y'' + b(t) \, y' + c(t) \, y = f(t),
            \]
            con \(t \in \mathrm{I}\), \(a\), \(b\), \(c\) e \(d\) funzioni continue in \(\mathrm{I}\) e \(a \neq 0\), allora,
            \(\forall \, t_0 \in \mathrm{I}\) e \(\forall \, (y_0, \, v_0) \in \R^2\), il problema di Cauchy
            \[
                \left\{
                    \begin{aligned}
                        &a(t) \, y'' + b(t) \, y' + c(t) \, y = f(t)\\
                        &y(t_0) = y_0\\
                        &y'(t_0) = v_0
                    \end{aligned}
                \right.
            \]
            ha \textbf{una ed una sola} soluzione definita su \textbf{tutto l'intervallo} \(\mathrm{I}\).
        \end{teorema}
            
\end{document}