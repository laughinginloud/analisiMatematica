% Imposto la radice del documento, utile per Visual Studio Code ed altri editor
%! TEX root = ../../analisi2.tex

% Imposto il file come sottofile del documento principale
\documentclass[../../analisi2]{subfiles}

\begin{document}

    \chapter{Introduzione alle equazioni differenziali lineari del secondo ordine}

        Un'equazione differenziale del secondo ordine si presenta nella forma
        \[
            a(t) \, y'' + b(t) \, y' + c(t) \, y = f(t),
        \]
        con \(t \in \mathrm{I}\). Definiamo dunque una sua soluzione.

        \begin{definizione}[Soluzione di un'equazione differenziale del secondo ordine]
            Si dice \textbf{soluzione dell'equazione differenziale} nell'intervallo \(\mathrm{I} \subset \R\) una funzione
            \(y : \mathrm{I} \to \R\) \textbf{derivabile due volte} per cui, sostituendo nell'equazione differenziale i valori effettivi
            di \(y(t)\), \(y'(t)\) e \(y''(t)\), si ottiene che
            \[
                a(t) \, y'' + b(t) \, y' + c(t) \, y = f(t) \quad \forall \, t \in \mathrm{I},
            \] cioè un'identità su \(\mathrm{I}\).
        \end{definizione}

        Un'equazione differenziale del secondo ordine ha soluzioni infinite. Queste vengono racchiuse nella loro totalità
        in dipendenza da due parametri all'interno dell'integrale generale. Se a questo aggiungiamo una coppia di condizioni iniziali
        otteniamo una soluzione specifica. Il sistema formato dall'integrale generale e le condizioni iniziali è detto
        \textbf{problema di Cauchy} ed il teorema che garantisce l'unicità della soluzione è detto \textbf{teorema di Cauchy}.

        \begin{teorema}[Teorema di Cauchy]
            Data l'equazione differenziale
            \[
                a(t) \, y'' + b(t) \, y' + c(t) \, y = f(t),
            \]
            con \(t \in \mathrm{I}\), \(a\), \(b\), \(c\) e \(d\) funzioni continue in \(\mathrm{I}\) e \(a \neq 0\), allora,
            \(\forall \, t_0 \in \mathrm{I}\) e \(\forall \, (y_0, \, v_0) \in \R^2\), il problema di Cauchy
            \[
                \left\{
                    \begin{aligned}
                        &a(t) \, y'' + b(t) \, y' + c(t) \, y = f(t)\\
                        &y(t_0) = y_0\\
                        &y'(t_0) = v_0
                    \end{aligned}
                \right.
            \]
            ha \textbf{una ed una sola} soluzione definita su \textbf{tutto l'intervallo} \(\mathrm{I}\).
        \end{teorema}

        Introduciamo dunque un importante teorema che sfrutta la linearità delle equazioni: il principio di sovrapposizione.

        \begin{teorema}[Principio di sovrapposizione]
            Se \(y_1\) è soluzione di \(a \, y'' + b \, y' + c \, y = f_1\) ed \(y_2\) è soluzione di
            \(a \, y'' + b \, y' + c \, y = f_2\), allora la funzione
            \[
                y(t) = \mathrm{C_1} \, y_1(t) + \mathrm{C_2} \, y_2(t)
            \]
            è soluzione di
            \[
                a \, y'' + b \, y' + c \, y = \mathrm{C_1} \, f_1 + \mathrm{C_2} \, f_2.
            \]
        \end{teorema}

        Prendiamo ora un'equazione differenziale omogenea, ovvero con \(f = 0\). Possiamo a questo punto notare che l'insieme
        \(\mathrm{S}\) delle soluzioni forma uno \textbf{spazio vettoriale} di dimensione due. Da questo ricaviamo il teorema di
        struttura.

        \begin{teorema}[Teorema di struttura]
            L'integrale generale di
            \[
                a(t) \, y'' + b(t) \, y' + c(t) \, y = 0,
            \]
            con \(a\), \(b\) e \(c\) continue su \(\mathrm{I}\) e \(a(t) \neq 0\), è dato da tutte le combinazioni lineari
            \[
                y(t) = \mathrm{C_1} \, y_1(t) + \mathrm{C_2} \, y_2(t) \quad \forall \, \mathrm{C_1}, \, \mathrm{C_2} \in \R,
            \]
            dove \(y_1\) ed \(y_2\) sono due \textbf{soluzioni linearmente indipendenti} dell'equazione stessa.
        \end{teorema}

        Osserviamo adesso un'equazione con \(f \neq 0\). Questa è detta \text{completa} o \textbf{non omogenea}. Se poniamo \(f = 0\),
        otteniamo dunque l'equazione \textbf{omogenea associata}. Definendo \(y_P\) come una particolare soluzione dell'equazione
        completa e \(y_0\) come una particolare soluzione dell'equazione omogenea associata, notiamo che queste soddisfano il
        principio di sovrapposizione. Con queste premesse possiamo dunque estendere il teorema di struttura alle equazioni non
        omogenee.

        \begin{teorema}[Teorema di struttura per equazioni complete]
            L'integrale generale di
            \[
                a(t) \, y'' + b(t) \, y' + c(t) \, y = f(t),
            \]
            con \(a\), \(b\), \(c\) e \(f\) continue su \(\mathrm{I}\) e \(a(t) \neq 0\), è dato da \textbf{tutte e sole} le funzioni
            \[
                y(t) = \mathrm{C_1} \, y_1(t) + \mathrm{C_2} \, y_2(t) + y_P(t) \quad \forall \, \mathrm{C_1}, \, \mathrm{C_2} \in \R,
            \]
            dove \(y_1\) ed \(y_2\) sono due \textbf{soluzioni linearmente indipendenti} dell'equazione omogenea associata
            \[
                a(t) \, y'' + b(t) \, y' + c(t) \, y = 0
            \]
            e \(y_P\) è una \textbf{soluzione particolare} dell'equazione completa
            \[
                a(t) \, y'' + b(t) \, y' + c(t) \, y = f(t).
            \]
        \end{teorema}
            
\end{document}