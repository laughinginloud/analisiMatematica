% Imposto la radice del documento, utile per Visual Studio Code ed altri editor
%! TEX root = ../../analisi2.tex

% Imposto il file come sottofile del documento principale
\documentclass[../../dimostrazioni]{subfiles}

\begin{document}

    \chapter{Introduzione alle equazioni differenziali del primo ordine}

        Introduciamo il concetto di equazione differenziale.

        \begin{definizione}[Equazione differenziale ordinaria del primo ordine]
            Un'equazione differenziale ordinaria del primo ordine, per brevità EDO, è una relazione che coinvolge una funzione incognita
            \(y(x)\), dove \(x \in \mathbb{R}\), e la sua derivata prima \(y'(x)\):
            \[
                F\left(x, \, y(x), \, y'(x)\right) = 0.
            \]

            In altre parole, \textbf{una EDO è un'equazione nella quale l'incognita non è un numero, ma una funzione}.
        \end{definizione}

        \begin{definizione}[Forma normale di una EDO]
            Una EDO del primo ordine in forma normale è una EDO nella forma
            \[
                y'(x) = f\left(x, \, y(x)\right).
            \]
        \end{definizione}

        \begin{definizione}[Integrale generale e particolare di una EDO]
            Data la EDO
            \[
                F\left(x, \, y(x), \, y'(x)\right) = 0,
            \]
            chiamiamo \textbf{integrale generale} dell'equazione, più raramente soluzione generale, l'insieme di tutte le sue soluzioni.

            Si chiama \textbf{integrale particolare} dell'equazione, più raramente soluzione particolare, una specifica soluzione.
        \end{definizione}

        Alcune osservazioni:
        \begin{itemize}
            \item Nel caso particolare di EDO del tipo
                \[
                    y'(x) = f(x),
                \]
                cioè EDO del primo ordine, in forma normale con \(f\left(x, \, \xcancel{y(x)} \, \right)\), basta integrare:
                \[
                    y(x) = \int \! f(x) \, \mathrm{d}x.
                \]
            \item Più in generale, risolvere una EDO non significa calcolare un integrale, ma comunque trovare \(y\) conoscendo delle
                informazioni relative a \(y'\). Da qui il nome \emph{integrale generale}.
            \item In generale, una EDO ha infinite soluzioni, proprio come la soluzione di un integrale indefinito.
        \end{itemize}
            
\end{document}