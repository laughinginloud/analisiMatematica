% Imposto la radice del documento, utile per Visual Studio Code ed altri editor
% !TEX root = ../analisi2.tex

% Comando per l'indentazione degli elementi di un elenco
\newcommand{\indentitem}{\setlength\itemindent{25pt}}

% Comandi per le notazioni insiemistiche
\newcommand{\insieme}[1]{\overline{\rm \underline{#1}}}
\newcommand{\insiemeparti}[1]{\mathcal{P}\left(\insieme{#1}\right)}
\newcommand{\card}[1]{\left| #1 \right|}

% Evito che PGF vada in modalità legacy
\pgfplotsset{compat=1.16}

% Definisco le punte stondate in entrambi i versi [(-) e )-(]
\pgfarrowsdeclare{(}{)}{}
{
  \pgfsetroundcap{}
  \pgfpathmoveto{\pgfpoint{-1.5pt}{-1.5pt}}
  \pgfpatharc{90}{270}{-1.5pt}
  \pgfusepathqstroke{}
}
\pgfarrowsdeclare{)}{(}{}
{
  \pgfsetroundcap{}
  \pgfpathmoveto{\pgfpoint{1.5pt}{1.5pt}}
  \pgfpatharc{90}{270}{1.5pt}
  \pgfusepathqstroke{}
}

% Imposto gli ambienti per i teoremi
\newtheorem{teorema}{Teorema}
\newtheorem{corollario}{Corollario}
\newtheorem{lemma}{Lemma}
\newtheorem{definizione}{Definizione}

% Definizione di un comando veloce per scrivere i reali
\newcommand{\R}{\mathbb{R}}