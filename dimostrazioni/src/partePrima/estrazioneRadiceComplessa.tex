% Imposto la radice del documento, utile per Visual Studio Code ed altri editor
%! TEX root = ../../dimostrazioni.tex

% Imposto il file come sottofile del documento principale
\documentclass[../../dimostrazioni]{subfiles}

\begin{document}

    \chapter{Estrazione di radice complessa}

        Dati due numeri, \(z, z^* \in \mathbb{C}\), definiti come \(\sqrt[n]{z^*} = z\) vogliamo estrarre la radice
        di \(z^*\), quindi calcolare \(z^* = z^n\).

        Partiamo definendo i numeri:
        \begin{align*}
            z &= \rho (\cos \vartheta + i \sin \vartheta)\\
            z^* &= \rho^* (\cos \vartheta^* + i \sin \vartheta^*)
        \end{align*}

        Dall'uguaglianza \(z^* = z^n\) possiamo quindi ricavare
        \[
            \rho^* (\cos \vartheta^* + i \sin \vartheta^*) = \rho^n (\cos n\vartheta + i \sin n\vartheta)
        \]
        Questa uguaglianza è valida a meno di multipli di \(2\pi\).

        Possiamo quindi ricavare il sistema
        \[
            \left\{
                \begin{aligned}
                    \rho^* &= \rho^n\\
                    \vartheta_k &= k \times 2\pi = n\vartheta
                \end{aligned}
            \right.
            \qquad \text{\emph{da cui si ricava}} \qquad
            \left\{
                \begin{aligned}
                    \rho &= \sqrt[n]{\rho^*}\\
                    \vartheta &= \frac{\vartheta^*}{n} + k \frac{2\pi}{n}
                \end{aligned}
            \right.
        \]

\end{document}