% Imposto la radice del documento, utile per Visual Studio Code ed altri editor
%! TEX root = ../../dimostrazioni.tex

% Imposto il file come sottofile del documento principale
\documentclass[../dimostrazioni]{subfiles}

\begin{document}

    \chapter{Binomio di Newton}

        \section*{Enunciato}

            Il binomio di Newton è
            \[ (a + b)^n = \sum_{k = 0}^n \binom{n}{k} \, a^k \, b^{n - k} \]

        \section*{Dimostrazione}

            Per dimostrare l'enunciato, procediamo con una dimostrazione per induzione.

            % Nota: sono attaccate per evitare multiple righe vuote
            Dimostriamo l'enunciato per \(n = 0\):
            \begin{align*}
                (a + b)^0 \, &= \, \sum_{k = 0}^0 \binom{n}{k} \, a^k \, b^{n - k}\\
                        1 \, &= \, \binom{0}{0} a^0 \, b^0\\
                        1 \, &= \, 1
            \end{align*}

            Possiamo perciò considerare l'enunciato vero al passo \(n\).

            % Nota: si veda sopra
            Dimostriamolo per \(n + 1\):
            \begin{align*}
                (a + b)^{n + 1} \, &= \, (a + b) (a + b)^n\\
                                   &= \, (a + b) \sum_{k = 0}^n \binom{n}{k} \, a^k \, b^{n - k}\\
                                   &= \, \sum_{k = 0}^n \binom{n}{k} \, a^{k + 1} \, b^{n - k} + \sum_{k = 0}^n \binom{n}{k} \, a^k \, b^{n - k + 1}\\
                                   &= \, \sum_{k = 1}^{n + 1} \binom{n}{k - 1} \, a^k \, b^{n - k + 1} + \sum_{k = 0}^n \binom{n}{k} \, a^k \, b^{n - k + 1}\\
                                   &= \, \binom{n}{n} \, a^{n + 1} + \binom{n}{0} \, b^{n + 1} + \sum_{k = 1}^n \binom{n}{k - 1} \, a^k \, b^{n - k + 1} + \sum_{k = 1}^n \binom{n}{k} \, a^k \, b^{n - k + 1}\\
                                   &= \, \binom{n + 1}{n + 1} \, a^{n + 1} + \binom{n + 1}{0} \, b^{n + 1} + \sum_{k = 1}^n \binom{n + 1}{k} \, a^k \, b^{n - k + 1}\\
                                   &= \, \sum_{k = 0}^{n + 1} \binom{n + 1}{k} \, a^k \, b^{n + 1 - k}
            \end{align*}

            Abbiamo quindi dimostrato il binomio di Newton.

\end{document}