% Imposto la radice del documento, utile per Visual Studio Code ed altri editor
%! TEX root = ../../dimostrazioni.tex

% Imposto il file come sottofile del documento principale
\documentclass[../../dimostrazioni]{subfiles}

\begin{document}

    \chapter{Irrazionalità di \texorpdfstring{\(\sqrt{2}\)}{radice di due}}

        \section*{Enunciato}

            \(\sqrt{2}\) è un numero irrazionale. È pertanto impossibile scriverlo come frazione.

        \section*{Dimostrazione}

            Dimostriamo l'enunciato per assurdo.
            Partiamo quindi prendendo due numeri coprimi \(p, q \in \mathbb{R}\) tali che
            \begin{align*}
                \frac{p}{q} &= \sqrt{2}\\
                p &= \sqrt{2} q\\
                p^2 &= 2 q^2
            \end{align*}
            Abbiamo quindi tre casi:
            \begin{description}
                \item[\(p, q\) dispari:]
                    \(p\) sarà uguale a
                    \[
                        p = 2^0 \times \dots = p^2
                    \]
                    \(q\) sarà uguale a
                    \begin{align*}
                        q &= 2^0 \times \dots = q^2\\
                        2q &= 2^1 \times \dots
                    \end{align*}
                    Perciò, \(p^2 \neq 2q^2\).
                \item[\(p\) dispari, \(q\) pari:]
                    \(p\) sarà uguale a
                    \[
                        p = 2^0 \times \dots = p^2
                    \]
                    \(q\) sarà uguale a
                    \begin{align*}
                        q &= 2^k \times \dots\\
                        q^2 &= 2^{2k} \times \dots\\
                        2q &= 2^{2k + 1} \times \dots
                    \end{align*}
                    Perciò, \(p^2 \neq 2q^2\).
                \item[\(p\) pari, \(q\) dispari:]
                    \(p\) sarà uguale a
                    \begin{align*}
                        p &= 2^k \times \dots\\
                        p^2 &= 2^{2k} \times \dots
                    \end{align*}
                    \(q\) sarà uguale a
                    \begin{align*}
                        q &= 2^0 \times \dots = q^2\\
                        2q &= 2^1 \times \dots
                    \end{align*}
                    Perciò, \(p^2 \neq 2q^2\).
            \end{description}
            c.v.d.
\end{document}