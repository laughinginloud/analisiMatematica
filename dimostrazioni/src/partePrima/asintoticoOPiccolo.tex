% Imposto la radice del documento, utile per Visual Studio Code ed altri editor
%! TEX root = ../../dimostrazioni.tex

% Imposto il file come sottofile del documento principale
\documentclass[../../dimostrazioni]{subfiles}

\begin{document}

    \chapter{Proprietà di \texorpdfstring{\(\sim\) ed \(o\)}{asintotico ed o-piccolo}}

        \section*{Teorema fondamentale che li lega}

            \subsection*{Enunciato}

                Per \(x \to x_0\)
                \[
                    f \sim q \iff f = g + o(g)
                \]

            \subsection*{Dimostrazione}

                Lo dimostreremo in entrambe le direzioni. Partiamo da sinistra.
                \begin{align*}
                    f &= gh & h \to 1\\
                    f - g &= gh - g\\
                    f - g &= g (h - 1) & H = h - 1 \to 0\\
                    f - g &= o(g)\\
                    f &= o(g) + g\\
                    f &\sim g
                \end{align*}

                Partiamo ora da destra.
                \begin{align*}
                    f - g &= o(g)\\
                    f - g &= gh & h \to 0\\
                    f &= g + gh\\
                    f &= g (h + 1) & H = h + 1 \to 1\\
                    f &\sim g
                \end{align*}
                
                c.v.d.

        \section*{Proprietà di \(o\)}

            \begin{description}[style=nextline]
                \item[\(o(k \, g) = k \, o(g) = o(g)\)]
                    \textbf{Dimostrazione}
                    \begin{align*}
                        f &= o(k \, g)\\
                        f &= g \, kh & h \to 0 \therefore kh \to 0\\
                        f &= o(g)
                    \end{align*}
                \item[\(o(g) \pm o(g) = o(g)\)]
                    \textbf{Dimostrazione per \(o(g) + o(g) = o(g)\)}
                    \[
                        o(g) + o(g) = 2 o(g) = o(g)
                    \]
                    \textbf{Dimostrazione per \(o(g) - o(g) = o(g)\)}
                    \[
                        o(g) - o(g) = o(g) + (-o(g)) = o(g) + o(g) = o(g)
                    \]
                \item[\(f \, o(g) = o(f \, g)\)]
                    \textbf{Dimostrazione}
                    \begin{align*}
                        F &= o(g)\\
                        F &= gh & h \to 0\\
                        f \, F &= f \, gh\\
                        f \, o(g) &= o(f \, g)
                    \end{align*}
                \item[\({(o(g))}^n = o(g^n) \quad\text{\emph{con}} n \in \mathbb{R}^+\)]
                    \textbf{Dimostrazione}
                    \begin{align*}
                        G &= o(g)\\
                        G &= gh & h \to 0\\
                        G^n &= g^n h^n & H = h^n \to 0\\
                        {(o(g))}^n &= o(g^n)
                    \end{align*}
            \end{description}

        \section*{Proprietà di \(\sim\)}
        
            \begin{description}[style=nextline]
                \item[Se \(f \sim g\) allora \(f^n \sim g^n \quad \text{\emph{dove }} n \neq 0\)]
                    \textbf{Dimostrazione}
                    \begin{align*}
                        f &= g \, h & h \to 1\\
                        f^n &= g^n \, h^n & H = h^n \to 1
                    \end{align*} 
                \item[Se \(f_1 \sim g_1\) e \(f_2 \sim g_2\) allora \(f_1 \, f_2 \sim g_1 \, g_2\) e \(\frac{f_1}{f_2} \sim \frac{g_1}{g_2}\)] 
                    \textbf{Dimostrazione}
                    \begin{align*}
                        f_1 \sim g_1 \qquad f_1 &= g_1 \, h_1 & h_1 \to 1\\
                        f_2 \sim g_2 \qquad f_2 &= g_2 \, h_2 & h_2 \to 1\\
                        f_1 \, f_2 &= g_1 \, g_2 \, H & H = h_1 \, h_2 \to 1\\
                        \frac{f_1}{f_2} &= \frac{g_1}{g_2} \, H & H = \frac{h_1}{h_2} \to 1
                    \end{align*}
            \end{description}
            
\end{document}