% Imposto la radice del documento, utile per Visual Studio Code ed altri editor
%! TEX root = ../../dimostrazioni.tex

% Imposto il file come sottofile del documento principale
\documentclass[../../dimostrazioni]{subfiles}

\begin{document}

    \chapter{Teorema fondamentale delle successioni monotone}

        Si dimostrano diversi casi:
        \begin{enumerate}
            \indentitem \item una successione monotona e limitata \emph{converge};
            \indentitem \item una successione monotona e illimitata
                \begin{itemize}
                    \item \emph{diverge positivamente}, se crescente
                    \item \emph{diverge negativamente}, se decrescente;
                \end{itemize}
        \end{enumerate}            


        \section*{Primo caso}
            
            \subsection*{Enunciato}
                
                \subsubsection*{Ipotesi}
                         
                Fisso per comodità \(a_n\) monotona crescente

                \begin{enumerate}
                    \indentitem \item \(a_n \leqslant a_{n+1}\) (ipotesi di monotonia);
                    \indentitem \item \(a_n\ \in B_r(0)) con r > 0 \) (ipotesi di limitatezza);
                \end{enumerate}
                
                \subsubsection*{Tesi}
                    \[\lim_{n \to +\infty} a_n = L\]
            
            \subsection*{Dimostrazione}
                Essendo \(a_n\) limitata avrà un maggiorante in \(r\) e quindi avrà il \(\sup\).
                Dimostro che \(L = \sup{a_n}\)

                Definiamo un intorno \(\epsilon\) di \(\sup\) come \(B_\epsilon (S)\), allora
                \(\forall \epsilon \geqslant 0 \qquad S-\epsilon\) non è più sup (né maggiorante), quindi

                \[
                    \exists a_n^* \mid S-\epsilon < a_n^* \leqslant S \\
                    \forall n > n^* \qquad S-\epsilon < a_n^* \leqslant a_n \leqslant S \\
                    a_n \in B_\epsilon(S) 
                \]                
                
        \section*{Secondo caso}
            \subsection*{Enunciato}
            Consideriamo il caso della crescenza
                
            \subsubsection*{Ipotesi}
                 
            Fisso per comodità \(a_n\) monotona crescente

            \begin{enumerate}
               \indentitem \item \(a_n \leqslant a_{n+1}\) (ipotesi di monotonia);
               \indentitem \item \(\nexist B_r(0) > {a_n} \forall n \) (ipotesi di illimitatezza);
            \end{enumerate}
        
            \subsubsection*{Tesi}
                \[\lim_{n \to +\infty} a_n = +\infty\]
    
        \subsection*{Dimostrazione}
            Essendo \(a_n\) illimitata 

            \[\forall B_r(0) a_n^* \geqslant r \]

            e per la monotonia

            \[\forall n > n^* \quad a_n^* \leqslant a_n\]

            quindi, definitivamente

            \[a_n \in B_r(+\infty)\]

            e per la definizione di limite 

            \[lim_{n \to +\infty} a_n = +\infty\]


\end{document}