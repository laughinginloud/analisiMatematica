% Imposto la radice del documento, utile per Visual Studio Code ed altri editor
%! TEX root = ../../dimostrazioni.tex

% Imposto il file come sottofile del documento principale
\documentclass[../dimostrazioni]{subfiles}

\begin{document}

    \chapter{Formule di De Morgan}

        \section*{Prima formula}

            La prima formula è
            \[
                (A \cup B)^\complement \equiv A^\complement \cap B^\complement
            \]

            La dimostreremo in entrambe le direzioni:
            \begin{itemize}
                \item se \(x \in (A \cup B)^\complement\), allora \(x \notin A \cup B\), quindi
                    \(x \notin A \, \land \, x \notin B\), perciò \(x \in A^\complement \, \land \, x \in B^\complement\),
                    dunque \(x \in A^\complement \cap B^\complement\);
                \item se \(x \in A^\complement \cap B^\complement\), allora \(x \in A^\complement \land x \in B^\complement\),
                    quindi \(x \notin A \land x \notin B\), perciò \(x \notin A \cup B\), dunque \(x \in (A \cup B)^\complement\).
            \end{itemize}

            Abbiamo quindi dimostrato la prima formula.

        \section*{Seconda formula}

            La seconda formula è
            \[
                (A \cap B)^\complement \equiv A^\complement \cup B^\complement
            \]

            Anche questa la dimostreremo in entrambe le direzioni:
            \begin{itemize}
                \item se \(x \in (A \cap B)^\complement\), allora \(x \notin A \cap B\), quindi
                    \(x \notin A \, \lor \, x \notin B\), perciò \(x \in A^\complement \, \lor \, x \in B^\complement\),
                    dunque \(x \in A^\complement \cup B^\complement\);
                \item se \(x \in A^\complement \cup B^\complement\), allora \(x \in A^\complement \lor x \in B^\complement\),
                    quindi \(x \notin A \lor x \notin B\), perciò \(x \notin A \cap B\), dunque \(x \in (A \cap B)^\complement\).
            \end{itemize}

            Abbiamo quindi dimostrato anche la seconda formula.

\end{document}