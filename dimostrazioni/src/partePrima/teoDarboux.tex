% Imposto la radice del documento, utile per Visual Studio Code ed altri editor
%! TEX root = ../../dimostrazioni.tex

% Imposto il file come sottofile del documento principale
\documentclass[../../dimostrazioni]{subfiles}

\begin{document}

    \chapter{Teorema di Darboux}
    \label{teoDarboux}

        \section*{Enunciato}

            \subsection*{Ipotesi}

                Sia \(f(x)\) una funzione tale che
                \begin{align*}
                    f : A \subseteq \mathbb{R} &\longrightarrow \mathbb{R}\\
                    x &\longmapsto y = f(x) 
                \end{align*}

                Supponiamo inoltre che:

                \begin{enumerate}
                    \indentitem \item \(A = [a, b]\) è un intervallo compatto;
                    \indentitem \item \(f\) è continua su \(A\).
                \end{enumerate}

            \subsection*{Tesi}
            
                \[\forall \, \lambda \; | \; m < \lambda < M, \, \exists \, x_\lambda \in A \; | \; f(x_\lambda) = \lambda\]

        \section*{Dimostrazione}

            Valendo il teorema di Weirstrass sappiamo che esistono \(M\), \(m\), \(x_M\) e \(x_m\) tali che
            \[
                f(x_m) = m \leqslant f(x) \leqslant M = f(x_M)
            \]

            Introduciamo quindi una funzione ausiliaria
            \[
                g(x) = f(x) - \lambda
            \]

            Notare come \(g\) ha la stessa regolarità di \(f\), infatti è continua. Inoltre \(g\), studiata
            nell'intervallo \([x_m, x_M]\) soddisfa il \hyperref[teoBolzano]{teorema di Bolzano}. Dunque,
            \begin{align*}
                g(x_m) = f(x_m) - \lambda = m - \lambda &< 0\\
                g(x_M) = f(x_M) - \lambda = M - \lambda &> 0
            \end{align*}
            da cui
            \begin{align*}
                \exists \, x_\lambda \; | \; &g(x_\lambda) = 0\\
                &f(x_\lambda) - \lambda = 0\\
                &f(x_\lambda) = \lambda
            \end{align*}
        
\end{document}