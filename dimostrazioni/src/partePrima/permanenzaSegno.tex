% Imposto la radice del documento, utile per Visual Studio Code ed altri editor
%! TEX root = ../../dimostrazioni.tex

% Imposto il file come sottofile del documento principale
\documentclass[../../dimostrazioni]{subfiles}

\begin{document}

    \chapter{Teorema di permanenza del segno}
    \label{teoPermSegno}

        \section*{Enunciato}

            Se \(\{a_n\}\) è definitivamente positiva e convergente allora il suo limite sarà non negativo.

        \section*{Dimostrazione}

            Dalle ipotesi del teorema possiamo ricavare
            \begin{gather*}
                \exists \, M \, | \, \forall n > M, \, a_n > 0\\
                L = \lim_{n \to +\infty} a_n
            \end{gather*}

            Vogliamo dimostrare che \(L \geqslant 0\). Procediamo per assurdo supponendo \(L < 0\). Per la definizione
            di limite,
            \[
                \forall \, \mathrm{B}_r (L), \, \exists \, M^* \, | \, \forall n > M^*, \, a_n \in \mathrm{B}_r (L)
            \]

            Se definiamo \(r\) come
            \[
                r < \frac{|L|}{2}
            \]
            stiamo dicendo che la successione è definitivamente negativa da \(M^*\) in poi, il che è assurdo poiché
            contraddice l'ipotesi.
    
\end{document}