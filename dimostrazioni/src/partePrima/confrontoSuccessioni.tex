% Imposto la radice del documento, utile per Visual Studio Code ed altri editor
%! TEX root = ../../dimostrazioni.tex

% Imposto il file come sottofile del documento principale
\documentclass[../../dimostrazioni]{subfiles}

\begin{document}

    \chapter{Teorema del confronto}

        \section*{Enunciato}

            Siano \(\{a_n\}, \{b_n\}, \{c_n\}\) tali che definitivamente \(a_n \leqslant b_n \leqslant c_n\).
            Inoltre, \(\{a_n\}, \{c_n\}\) convergono a \(L\). Allora,
            \[
                L = \lim_{n \to +\infty} b_n
            \]

        \section*{Dimostrazione}

            Dalle ipotesi del teorema possiamo ricavare
            \begin{gather*}
                \exists \, M_1 \; | \; \forall n > M_1, \, a_n \leqslant b_n \leqslant c_n\\
                \forall \, \mathrm{B}_r (L), \, \exists \, M_2 \; | \; \forall n > M_2, \, a_n \in \mathrm{B}_r (L) \quad \text{\emph{ovvero}} \quad L - r < a_n < L + r\\
                \forall \, \mathrm{B}_r (L), \, \exists \, M_3 \; | \; \forall n > M_3, \, c_n \in \mathrm{B}_r (L) \quad \text{\emph{ovvero}} \quad L - r < c_n < L + r
            \end{gather*}

            Chiamiamo \(M^* = \max\{M_1, M_2, M_3\}\). Dopo \(M^*\) valgono le tre precedenti relazioni. Dunque,
            \[
                \forall n > M^*, \, L - r < a_n \leqslant b_n \leqslant c_n < L + r
            \]
            da cui si deduce che
            \[
                b_n \in \mathrm{B}_r (L)
            \]
            ovvero
            \[
                \lim_{n \to +\infty} b_n = L
            \]
    
\end{document}