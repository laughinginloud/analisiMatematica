% Imposto la radice del documento, utile per Visual Studio Code ed altri editor
%! TEX root = ../../dimostrazioni.tex

% Imposto il file come sottofile del documento principale
\documentclass[../../dimostrazioni]{subfiles}

\begin{document}

    \chapter{Formule di de Moivre}

        \section*{Prodotto di due numeri complessi}

            Dati due numeri complessi \(z_1\) e \(z_2\) definiti come
            \begin{align*}
                z_1 \, &= \, \rho_1(\cos\vartheta_1 + i\sin\vartheta_1)\\
                z_2 \, &= \, \rho_2(\cos\vartheta_2 + i\sin\vartheta_2)
            \end{align*}

            Il loro prodotto sarà uguale a
            \begin{align*}
                z_1 \, z_2 \, =& \, \rho_1(\cos\vartheta_1+i\sin\vartheta_1) \, \rho_2(\cos\vartheta_2+i\sin\vartheta_2)\\
                       =& \, \rho_1\rho_2 (\cos\vartheta_1\cos\vartheta_2 + \cos\vartheta_1i\sin\vartheta_2 + i\sin\vartheta_1\cos\vartheta_2 - \sin\vartheta_1\sin\vartheta_2) \\
                       =& \, \rho_1\rho_2 [(\cos\vartheta_1\cos\vartheta_2 - \sin\vartheta_1\sin\vartheta_2) + i(\cos\vartheta_1 \sin\vartheta_2 + \sin\vartheta_1\cos\vartheta_2)] \\
                       =& \, \rho_1\rho_2 [\cos(\vartheta_1 + \vartheta_2) + i\sin(\vartheta_1 + \vartheta_2)]
            \end{align*}

        \section*{Quoziente di due numeri complessi}

            \begin{align*}
                \frac{z_1}{z_2} \, =& \, \frac{\rho_1\cos\vartheta_1+i\sin\vartheta_1}{\rho_2\cos\vartheta_2+i\sin\vartheta_2} \\
                                =& \, \frac{\rho_1}{\rho_2} \frac{\cos\vartheta_1+i\sin\vartheta_1}{\cos\vartheta_2+i\sin\vartheta_2} \frac{\cos\vartheta_2-i\sin\vartheta_2}{\cos\vartheta_2-i\sin\vartheta_2}\\
                                =& \, \frac{\rho_1}{\rho_2} \frac{\cos\vartheta_1\cos\vartheta_2 - \cos\vartheta_1 i\sin\vartheta_2 + i\sin\vartheta_1\cos\vartheta_2 + \sin\vartheta_1\sin\vartheta_2}{\cos^2\vartheta_2 - i\sin\vartheta_2\cos\vartheta_2 + i\sin\vartheta_2\cos\vartheta_2 + \sin^2\vartheta_2} \\
                                =& \, \frac{\rho_1}{\rho_2} \frac{[\cos(\vartheta_1 - \vartheta_2) + i\sin(\vartheta_1 - \vartheta_2)]}{\cos^2\vartheta_2 + \sin^2\vartheta_2} \\
                                =& \, \frac{\rho_1}{\rho_2} [\cos(\vartheta_1 - \vartheta_2) + i\sin(\vartheta_1 - \vartheta_2)]
            \end{align*}

        \newpage

        \section*{Potenza di numero complesso}

            \subsection*{Enunciato}

                \[ z^n = \rho^n (\cos n\vartheta +i\sin n\vartheta)\]

            \subsection*{Dimostrazione}

                Lo dimostreremo per induzione.

                Dimostriamo l'enunciato per \(n = 2\):
                \begin{align*}
                    z^2 = z \, z =& \, \rho(\cos\vartheta+i\sin\vartheta) \, \rho(\cos\vartheta+i\sin\vartheta)\\
                                      =& \, \rho^2(\cos2\vartheta + i\sin2\vartheta)
                \end{align*}
                secondo la prima formula, con \(z_1 = z_2 = z\).
                Possiamo perciò considerare l'enunciato vero al passo \(n - 1\).

                Dimostriamolo per \(n\):
                \begin{align*}
                    z^{n - 1} \, =& \, \rho^{n - 1} (\cos (n - 1)\vartheta + i \sin (n - 1)\vartheta)\\
                    z^{n} \, = \, z^{n - 1} z \, =& \, \rho^{n - 1} (\cos (n - 1)\vartheta + i \sin (n - 1)\vartheta) \, \rho(\cos \vartheta + i \sin \vartheta) \\
                                               =& \, \rho^n (\cos n\vartheta + i\sin n\vartheta)
                \end{align*}
                sempre secondo la prima formula.

                Abbiamo perciò dimostrato anche la terza formula. c.v.d.
            
\end{document}