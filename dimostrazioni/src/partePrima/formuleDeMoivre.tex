% Imposto la radice del documento, utile per Visual Studio Code ed altri editor
%! TEX root = ../../dimostrazioni.tex

% Imposto il file come sottofile del documento principale
\documentclass[../../dimostrazioni]{subfiles}

\begin{document}

    \chapter{Formule di De Moivre}

        \section*{Prodotto di due numeri complessi}

            Dati due numeri complessi \(z_1\) e \(z_2\) definiti come
            \begin{align*}
                z_1 \, &= \, \rho_1(\cos\vartheta_1 + i\sin\vartheta_1)\\
                z_2 \, &= \, \rho_2(\cos\vartheta_2 + i\sin\vartheta_2)
            \end{align*}

            Il loro prodotto sarà uguale a
            \begin{align*}
                z_1z_2 =& [\rho_1\cos\vartheta_1+i\sin\vartheta_1\rho_2\cos\vartheta_2+i\sin\vartheta_2] \\
                       =& \rho_1\rho_2 [\cos\vartheta_1\cos\vartheta_2 + \cos\vartheta_1i\sin\vartheta_2 + i\sin\vartheta_1\cos\vartheta_2 - \sin\vartheta_1\sin\vartheta_2] \\
                       =& \rho_1\rho_2 [(\cos\vartheta_1\cos\vartheta_2 - \sin\vartheta_1\sin\vartheta_2) + (\cos\vartheta_1 i\sin\vartheta_2 + i\sin\vartheta_1\cos\vartheta_2)] \\
                       =& \rho_1\rho_2 (\cos(\vartheta_1 + \vartheta_2) + i\sin(\vartheta_1 + \vartheta_2))
            \end{align*}

        \section*{Quoziente di due numeri complessi}

            \begin{align*}
                \frac{z_1}{z_2} =& \frac{\rho_1\cos\vartheta_1+i\sin\vartheta_1}{\rho_2\cos\vartheta_2+i\sin\vartheta_2} \\
                                =& \frac{\rho_1}{\rho_2} \frac{\cos\vartheta_1+i\sin\vartheta_1}{\cos\vartheta_2+i\sin\vartheta_2} \frac{\cos\vartheta_1-i\sin\vartheta_1}{\cos\vartheta_2-i\sin\vartheta_2}\\
                                =& \frac{\rho_1}{\rho_2} \frac{\cos\vartheta_1\cos\vartheta_2 - \cos\vartheta_1 i\sin\vartheta_2 + i\sin\vartheta_1\cos\vartheta_2 + \sin\vartheta_1\sin\vartheta_2)}{\cos^2\vartheta_2 - i\sin\vartheta_2\cos\vartheta_2 + i\sin\vartheta_2\cos\vartheta_2 + \sin^2\vartheta_2} \\
                                =& \frac{\rho_1}{\rho_2} \frac{[\cos(\vartheta_1 - \vartheta_2) + i\sin(\vartheta_1 - \vartheta_2)]}{\cos^2\vartheta_2 + \sin^2\vartheta_2} \\
                                =& \frac{\rho_1}{\rho_2} [\cos(\vartheta_1 - \vartheta_2) + i\sin(\vartheta_1 - \vartheta_2)]
            \end{align*}

        \newpage

        \section*{Potenza di numero complesso}

            Si dimostra per induzione.
            
            \medskip

            Dimostriamo l'enunciato per \(n = 2\):
            \begin{align*}
                z^2 \, = \, zz \, =& \, \rho\rho [\cos(\vartheta + \vartheta) + i\sin(\vartheta+\vartheta)] \\
                                  =& \, \rho^2(\cos2\vartheta)+ i\sin2\vartheta)
            \end{align*}

            Che è quindi, in generale, uguale a

            \[\rho^n (\cos n\vartheta +i\sin n\vartheta)\]

            Possiamo perciò considerare l'enunciato vero al passo \(n\).

            \medskip

            Dimostriamolo per \(n + 2\):
            \begin{align*}
                z^{n+2} \, = \, z^n z^2 \, =& \, \rho^n \rho^2 [(\cos n\vartheta + i\sin n\vartheta)(\cos 2\vartheta + i\sin 2\vartheta)] \\
                                           =& \, \rho^n \rho^2 [\cos n\vartheta\cos2\vartheta + \cos n\vartheta i\sin2\vartheta + i\sin n\vartheta \cos2\vartheta - \sin n\vartheta \sin2\vartheta] \\
                                           =& \, \rho^{n+2} (\cos(n\vartheta + 2\vartheta) + i\sin(n\vartheta + 2\vartheta)) \\
                                           =& \, \rho^{n+2} (\cos(n+2)\vartheta + i\sin(n+2)\vartheta)
            \end{align*}
            

\end{document}