% Imposto la radice del documento, utile per Visual Studio Code ed altri editor
%! TEX root = ../dimostrazioni.tex

% Imposto il file come sottofile del documento principale
\documentclass[../dimostrazioni]{subfiles}

\begin{document}

    % Il comando serve a mettere il testo che andrà nella struttura del documento
    % Non è possibile usare Unicode, quindi è stato scelto "del continuo" al posto di R^2
    \chapter{Cardinalità \texorpdfstring{di \(\mathbb{R}^n\)}{del continuo}}

        \section*{Definizioni necessarie}

            Si ricorda che:

            \begin{itemize}
                \item Due insiemi hanno la stessa cardinalità quando è possibile creare una corrispondenza biunivoca tra di essi;
                \item Un insieme infinito può avere la stessa cardinalità di un insieme infinito da lui contenuto;
            \end{itemize}

        \section*{Enunciato}

            \subsection*{Ipotesi}

            \( \mathbb{R} \) ha la cardinalità del continuo.

            \subsection*{Tesi}

            \( \mathbb{R}^n \) ha la cardinalità del continuo.                

        \section*{Dimostrazione}

            Come definito in precedenza per dimostrare che i due insiemi hanno la stessa cardinalità 
            dobbiamo dimostrare che siano in corrispondenza \textbf{biunivoca}. 
            Per semplicità restringiamo la dimostrazione all'intervallo \([0, 1]\).
            
            \subsection*{Iniettività}

                Dato un punto generico \(P (x_P,y_P) \) definiamo che le sue coordinate in questo modo:
                \[  x_p = 0.x_1 \, x_2 \, x_3 \, x_4 \dots \, \, \text{e} \, \, y_p = 0.y_1 \, y_2 \, y_3 \, y_ 4 \dots \]

                L'immagine di \(P\) su \( \mathbb{R} \) è \( Q \), così definita:
                \[   Q = 0.x_1 \, y_1 \, x_2 \, y_2 \, x_3 \, y_3 \, x_4 \, y_ 4 \dots    \]

                Ipotizziamo ora per assurdo che esista 

                \[    P^* \neq P \, | \, f(P^*) = f(P) \]

                \[    P^* = (0.x^*_1 \, x^*_2 \, x^*_3 \, x^*_4 \dots , 0.y^*_1 \, y^*_2 \, y^*_3 \, y^*_4 \dots) \]

                allora

                \[    f(P^*) = Q = 0.x^*_1 \, y^*_1 \, x^*_2 \, y^*_2 \, x^*_3 \, y^*_3 \, x^*_4 \, y^*_4 \dots \]

                Ma visto che

                \[ Q = 0.x_1 \, y_1 \, x_2 \, y_2 \, x_3 \, y_3 \, x_4 \, y_ 4 \dots \]

                ne deriva che

                \[P = P^*\]

                il che è assurdo. Quindi \(f\) è \textbf{iniettiva}.

            \subsection*{Suriettività}

                Dato
                \[    Q \in [0,1] = 0.q_1 \, q_2 \, q_3 \, q_4 \dots    \]

                Vale questa affermazione?

                \[    \exists \text{?} \, P° \in [0,1] \times [0,1] \, | \, f(P°) = Q    \]

                Sì, \(P°\) è così definito:

                \[    P° = (0.q_1 \, q_3 \, q_5 \dots, 0.q_2 \, q_4 \, q_6 \dots)    \]

                Da cui si ricava che \(f\) è anche \textbf{suriettiva}.

                \bigskip

                Abbiamo quindi trovato una corrispondenza biunivoca tra i due insiemi, il che dimostra che hanno la stessa cardinalità.
            
\end{document}