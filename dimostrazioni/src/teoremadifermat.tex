% Imposto la radice del documento, utile per Visual Studio Code ed altri editor
%! TEX root = ../dimostrazioni.tex

% Imposto il file come sottofile del documento principale
\documentclass[../dimostrazioni]{subfiles}

\begin{document}

    \chapter{Teorema di Fermat}

        \section*{Enunciato}

            La disuguaglianza di Bernoulli è

            \[ (1 + x)^n \geqslant 1 + nx \qquad \forall n \in \mathbb{N}, \, \forall x \in \mathbb{R}, \, x > -1 \]

        \section*{Dimostrazione}

            Per dimostrare l'enunciato, procediamo con una dimostrazione per induzione.

            \medskip

            % Nota: sono attaccate per evitare multiple righe vuote
            Dimostriamo l'enunciato per \(n = 0\):
            \begin{align*}
                (1 + x)^0 \, &\geqslant \, 1 + 0 x\\
                        1 \, &\geqslant \, 1
            \end{align*}

            Possiamo perciò considerare l'enunciato vero al passo \(n\).

            \medskip

            % Nota: si veda sopra
            Dimostriamolo per \(n + 1\):
            \begin{align*}
                (1 + x)^{n + 1} \, =& \, (1 + x)(1 + x)^n\\
                                \geqslant& \, (1 + x)(1 + nx) & \text{\emph{Per ipotesi induttiva}}\\
                                =& \, 1 + nx + x + nx^2\\
                                =& \, 1 + x (n + 1) + nx^2\\
                                \geqslant& \, 1 + x (n + 1) & \text{\emph{Per l'enunciato del teorema}}
            \end{align*}

            Abbiamo quindi dimostrato la disuguaglianza di Bernoulli.
    
\end{document}