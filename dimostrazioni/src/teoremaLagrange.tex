% Imposto la radice del documento, utile per Visual Studio Code ed altri editor
%! TEX root = ../dimostrazioni.tex

% Imposto il file come sottofile del documento principale
\documentclass[../dimostrazioni]{subfiles}


\begin{document}

    \chapter{Teorema di Lagrange}

        \section*{Enunciato}

        \subsection*{Ipotesi}

            Sia \(f(x)\) una funzione tale che
            \begin{align*}
                f : A = [a, b] &\longrightarrow \mathbb{R}\\
                x &\longmapsto y = f(x) 
            \end{align*}

            Supponiamo inoltre che:

            \begin{enumerate}
                \indentitem \item \(f\) sia continua su \(A\) e derivabile su \((a, b)\);
            \end{enumerate}

        \subsection*{Tesi}

            \[ \exists \, x_0 \in (a,b) \, / \, f'(x_0) = \frac{f(b)-f(a)}{b-a}=m \]
            dove \(m\) è il coefficiente angolare della retta passante \(a\) e \(b\).

        \section*{Dimostrazione}

            \medskip

            Introduco una \textbf{funzione ausiliaria} \(g(x)\) così definita:

            \[ g(x) = f(x) - \bigg[f(a) + \frac{f(b) + f(a)}{b - a}(x - a) \bigg] \]

            Notiamo che \(g\) ha la regolarità di \(f\) su A
            \begin{enumerate}
                \indentitem \item è continua su \( A \);
                \indentitem \item derivabile su \( (a, b) \);
            \end{enumerate}

            Notiamo anche che:
            \begin{align*}
                g(a) &= f(a) - \bigg[f(a) + \frac{f(b) - f(a)}{b - a}(a - a) \bigg]\\
                     &= f(a) - \bigg[f(a) + 0 \bigg]\\
                     &= f(a) - f(a) = 0\\
                     \\
                g(b) &= f(b) - \bigg[f(a) + \frac{f(b) - f(a)}{b - a}(b - a) \bigg]\\
                     &= f(b) - \bigg[f(a) + f(b) - f(a) \bigg]\\
                     &= f(b) - f(b) = 0
            \end{align*}
            
            Da cui \( g(a) = g(b) \).

            Posso quindi applicare il teorema di \textbf{Rolle} su \( A \):
            \[
                \exists \, \, x_0 \in (a,b) \, \, / \, \, g'(x_0) = 0
            \]
            
            Calcolo quindi \( g'(x) \):
            \[
                g'(x) = f'(x) - \frac{f(b) - f(a)}{b-a}\\
                \\
                g'(x_0) = 0\\
                \\
                f'(x_0) - \frac{f(b) - f(a)}{b-a} = 0\\
                f'(x_0) = \frac{f(b) - f(a)}{b-a}
            \] c.v.d.
    
\end{document}