% Imposto la radice del documento, utile per Visual Studio Code ed altri editor
%! TEX root = ../dimostrazioni.tex

% Imposto il file come sottofile del documento principale
\documentclass[../dimostrazioni]{subfiles}

\begin{document}

    \chapter{Criterio della Radice per la Convergenza delle Serie a termini positivi.}
    \label{criterioRadiceSerie}

        \section*{Enunciato}

            Sia \(\sum_{n\dots}^{+\infty}\) \(a_n\) una serie a termini positivi \(a_n\) > 0 \(\forall n\). Se:
             \[\sqrt[n]{a_n} \longrightarrow l \qquad  per \; n  \rightarrow +\infty \]
                
            
            Allora:
            \[
                \left\{
                    \begin{aligned}
                        &se \; l > 1 & \text{diverge} \\
                        &se \; l = 1 & \text{il criterio non si applica} \\
                        &se \; 0 \leqslant l < 1 & \text{converge}
                    \end{aligned}
                \right.
            \]
            
        \section*{Dimostrazione}
            
            \subsection*{Caso 1:  \(0 \leqslant l < 1\)} 
                    
                Introduco una \textbf{successione ausiliaria}
                \[b_n = \sqrt[n]{a_n} \]  
                \[\lim_{n \to +\infty} b_n = l \qquad \text{e so che} l < 1\]
                Per la definizione di limite:
                \[\forall B_\varepsilon (l) \; \exists M \mid \forall n > M \quad b_n \in B_\varepsilon (l) \]
                Scelgo \(\varepsilon\) in modo che \(\varepsilon < 1 -l \)
                \begin{align*}
                    l - \varepsilon < b_n &< l + \varepsilon \; (<1) \\
                    \sqrt[n]{a_n} &< l + \varepsilon \; (<1) \\
                    a_n &< (l+\varepsilon)^{n}  
                \end{align*}
                                     
                Applico il \textbf{\hyperref[serieconfronto]{criterio del confronto}} tra \(\Sigma a_n\) e:
                \begin{align*}
                    && \text{geometrica di ragione:} \\
                    \sum_{n=\dots}^{+\infty} (l+\varepsilon)^{n} && q = l+ \varepsilon \\
                    && -1 < q < 1
                \end{align*}

                Essendo quest'ultima convergente, possiamo concludere che anche la serie di partenza \textbf{converge}.



            \subsection*{Caso 2: \(l > 1\)}




                        
\end{document}