% Imposto la radice del documento, utile per Visual Studio Code ed altri editor
%! TEX root = ../dimostrazioni.tex

% Imposto il file come sottofile del documento principale
\documentclass[../dimostrazioni]{subfiles}

\begin{document}

    \chapter{Condizione necessaria per la convergenza di una serie}
    \label{CNconvergenza}

        \section*{Definizioni necessarie}

            \begin{itemize}
                \item Data la successione:
                        \begin{align*}
                            a_n : \mathbb{N} &\longrightarrow \mathbb{R}\\
                            n &\longmapsto a_n 
                        \end{align*}
    
                        Si dice \textbf{serie}:

                        \[  \sum_{n=0}^{+\infty} a_n\]

                \item La successione \(a_n\) è detta \textit{argomento} della serie.
                \item La \textbf{successione delle somme parziali} è così definita:
                      
                        \[  S_N = \sum_{n=0}^{N} a_n\]
                    
                \item Il  \textit{carattere (o la natura)} della \textbf{serie} è il \textit{carattere (o la natura)} della sua \textbf{successione delle somme parziali}
            \end{itemize}
           

        \section*{Enunciato}

            \subsection*{Ipotesi}

                \[  \sum_{n=0}^{+\infty} a_n \qquad \text{converge} \]
                
            \subsection*{Tesi}

                \[  \lim_{n \to +\infty} a_n \rightarrow 0  \]

        \section*{Dimostrazione}

            Se \(\sum_{n=0}^{+\infty} a_n \) converge allora:

            \[  L = \lim_{N \to +\infty} S_N   \]

            \subsection*{Osservazione}

                Posso definire \(S_N\) ricorsivamente:

                \[  
                    \left\{
                        \begin{aligned}
                            S_{N+1} &= S_N + a_{N+1}\\
                            S_0 &= a_0
                        \end{aligned}
                    \right.
                \]

            Noto che anche:

            \[  \lim_{N \to +\infty} S_{N+1} = L \]

            \[  \lim_{N \to +\infty} S_{N} = L \]
            
            Essendo i due limiti finiti posso fare il limite della loro differenza:

            \[  \lim_{N \to +\infty}\left(S_{N+1} - S_N \right) = L - L = 0  \]

            Dalla definizione ricorsiva che ho dato di \(S_N\) posso riscrivere il tutto come:

            \[  \lim_{N \to +\infty}\left(S_{N+1} - S_N \right) = \lim_{N \to +\infty} \left(S_N + a_{N+1} - S_N \right) = \lim_{N \to +\infty} a_{N+1} \]

            Da quanto sopra sappiamo che \(  \lim_{N \to +\infty}\left(S_{N+1} - S_N \right) \rightarrow 0  \) quindi:

            \[  \lim_{N \to +\infty}\left(S_{N+1} - S_N \right) = \lim_{N \to +\infty} a_{N+1} \rightarrow 0 \]

            c.v.d.
\end{document}