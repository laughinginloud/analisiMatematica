% Imposto la radice del documento, utile per Visual Studio Code ed altri editor
%! TEX root = ../../dimostrazioni.tex

% Imposto il file come sottofile del documento principale
\documentclass[../../dimostrazioni]{subfiles}

\begin{document}

    \chapter{Criterio del confronto per la convergenza di una serie a termini positivi}
    \label{serieconfronto}

        \section*{Enunciato}

            \subsection*{Ipotesi}

                Siano

                \[  \sum_{n=\dots}^{+\infty} a_n \quad \text{e} \quad  \sum_{n=\dots}^{+\infty} b_n   \]

                tali che:
                \begin{enumerate}
                    \indentitem \item \(\exists \, M_1 \mid \forall \, n \geqslant M_1\,,\, a_n > 0 \land b_n > 0  \);
                    \indentitem \item \(\exists \, M_2 \mid \forall \, n \geqslant M_2\,,\, a_n \leqslant b_n \).
                \end{enumerate}
                
            \subsection*{Tesi}
                \begin{enumerate}
                    \indentitem \item Se \(\sum_{n=\dots}^{+\infty} a_n \; \text{diverge, allora anche} \; \sum_{n=\dots}^{+\infty} b_n \; \text{diverge} \);
                    \indentitem \item Se \(\sum_{n=\dots}^{+\infty} b_n \; \text{converge, allora anche} \; \sum_{n=\dots}^{+\infty} a_n \; \text{converge} \).
                \end{enumerate}
                

        \section*{Dimostrazione}

            \subsection*{Parte 1 - Divergenza}

                Siano \(A_N = \sum_{n=\dots}^{N} a_n\) e \(B_N = \sum_{n=\dots}^{N} b_n\). 
                Se \(\sum_{n=\dots}^{+\infty} a_n\) diverge significa che \(\lim_{N \to +\infty} A_N = +\infty\) 
                quindi per definizione di limite:

                \[\forall \, B_r(+\infty) \, \exists \, R \mid \forall \, N > R  \quad A_N > R\]

                Ricordiamo che:
                \[ a_n \leqslant b_n \qquad (\forall \, n > M_1) \]

                Con le sommatorie:
                \[ \sum_{n = max(M_1,M_2)}^{+\infty} a_n \leqslant \sum_{n = max(M_1,M_2)}^{+\infty} b_n\]
                \[  A_N \leqslant B_N   \]

                Da cui:
                \[  \lim_{N \to +\infty} B_N = +\infty  \]
                \[  B_N > R \Rightarrow \; \text{Quindi} \; \sum b_n \text{diverge a} +\infty \]

                c.v.d.

            \subsubsection*{Parte 2 - Convergenza}

                Se \(\sum_{n=\dots}^{+\infty} b_n\) converge significa che \(   \lim_{N \to +\infty} B_N = L  \) ovvero per definizoone di limite:

                \[  \forall \, B_r(L)\, \exists \, M_3 \mid \forall \, n>M_3 \quad B_N \in B_r(L) \quad  L-r \leqslant B_N \leqslant L + r   \]
                
                \(A_N \leqslant B_N\) inoltre \(A_N, B_N\) sono monotone, infatti:
                \[  A_{N+1} = A_N + a_{N+1}\; \text{e} \; a_{N+1} > 0 \quad \text{perciò} \quad A_{N+1} > A_N\]

                \(A_N\) è strettamente crescente e limitata (dal valore di \(L\)).
                \[  A_N \leqslant B_N \leqslant L \]

                quindi per il teorema fondamentale delle successioni monotone \(A_N\) converge, c.v.d.

\end{document}