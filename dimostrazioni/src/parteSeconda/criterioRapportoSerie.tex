% Imposto la radice del documento, utile per Visual Studio Code ed altri editor
%! TEX root = ../../dimostrazioni.tex

% Imposto il file come sottofile del documento principale
\documentclass[../../dimostrazioni]{subfiles}

\begin{document}

    \chapter{Criterio del rapporto per la convergenza delle serie a termini positivi}
    \label{criterioRapportoSerie}

        \section*{Enunciato}

            Sia \(\sum a_n\) una serie a termini positivi, con \(a_n > 0, \, \forall \, n\). Se
             \[\frac{a_{n+1}}{a_n} \longrightarrow l \qquad  \text{per} \; n \rightarrow +\infty \]
            
            Allora
            \[
                l \;
                \begin{cases}
                    \text{diverge} & \text{se} \; l > 1\\
                    \text{il criterio non si applica} & \text{se} \; l = 1\\
                    \text{converge} & \text{se} \; 0 \leqslant l < 1
                \end{cases}
            \]
            
        \section*{Dimostrazione}
            
            \subsection*{Caso 1 - \(0 \leqslant l < 1\)}

                Introduco una \textbf{successione ausiliaria}
                \begin{gather*}
                    b_n = \frac{a_{n+1}}{a_n}\\
                    \lim_{n \to +\infty} b_n = l \qquad \text{e so che} \; l < 1
                \end{gather*}

                Per la definizione di limite
                \[\forall \, \mathrm{B}_\varepsilon (l), \, \exists \, M \mid \forall \, n > M, \, b_n \in \mathrm{B}_\varepsilon (l) \]

                Scegliamo \(\varepsilon\) in modo che \(\varepsilon < 1 - l\) da \(M\) in poi. Dunque,
                \begin{align*}
                    \frac{a_{n+1}}{a_n} &= b_n < l + \varepsilon \\
                    a_{n+1} &< a_n(l + \varepsilon) && \text{\emph{Disuguaglianza ricorsiva che vale \textbf{definitivamente}}} \\
                    a_{M+2} &< a_{M+1} (l + \varepsilon) \\
                    a_{M+3} &< a_{M+2} (l + \varepsilon) < a_{M+1}(l + \varepsilon)^{2}\\
                    a_{M+4} &< a_{M+3} (l + \varepsilon) < a_{M+1}(l + \varepsilon)^{3} \\
                    &\dots \\
                    a_{M+n+1} &< a_{M+1}(l + \varepsilon)^n
                \end{align*}

                Ho \textbf{maggiorato} definitivamente la serie di partenza con una serie
                \[ \sum_{n=\dots}^{+\infty} a_{M+1}(l + \varepsilon)^{n}\] 
                Applico il \textbf{\hyperref[serieconfronto]{criterio del confronto}} con la geometrica con ragione
                \[-1 < q = l + \varepsilon < 1 \]
                che \emph{converge}, quindi anche la serie di partenza \(\sum a_n \) \textbf{converge}, c.v.d.

            \subsection*{Caso 2 - \(l > 1\)}
                
                Definiamo una successione ausiliaria \(b_n\) come
                \[b_n = \frac{a_{n+1}}{a_n}\]

                Sappiamo inoltre che
                \[ \lim_{n \to +\infty}b_n = l > 1 \]

                Da ciò si può dedurre che
                \[\forall \, n \quad a_{n + 1} > a_n\]

                Da questo possiamo stabilire che la successione \(b_n\) è monotona strettamente crescente. Ma,
                poiché la successione \(b_n\) è stata definita in funzione della successione \(a_n\), significa
                che anche questa ultima è monotona strettamente crescente. In tal caso, significa che la serie ad
                essa associata \(\sum a_n\) diverge a \(+\infty\), c.v.d.
                
\end{document}