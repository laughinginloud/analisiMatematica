% Imposto la radice del documento, utile per Visual Studio Code ed altri editor
%! TEX root = ../../dimostrazioni.tex

% Imposto il file come sottofile del documento principale
\documentclass[../dimostrazioni]{subfiles}

\begin{document}

    \chapter{Teorema di Cauchy}
    \label{teoCauchy}

        \section*{Enunciato}

            \subsection*{Ipotesi}

                Date:
                \begin{align*}
                    f,g : A = [a, b] \longrightarrow \, &\mathbb{R}\\
                    x \longmapsto \, &y = f(x)\\
                                  &y = g(x) 
                \end{align*}

                Supponendo inoltre \(f, g\) continue in A e derivabili in \( (a,b) \).

            \subsection*{Tesi}

                \[ 
                    \exists \, x^* \in (a,b) \mid \frac{ f'(x^*) }{ g'(x^*) } = \frac{ f(b) - f(a) }{ g(b) - g(a) }
                \]

        \section*{Dimostrazione}

            Introduco una \textbf{funzione ausiliaria} \(h(x)\) così definita:

            \[ 
                h(x) = \left[f(b) - f(a)\right]g(x) - \left[g(b) - g(a)\right]f(x)
            \]

            Notiamo che \(h\) ha la regolarità di \(f\) e di \(g\) su A:
            \begin{enumerate}
                \indentitem \item è continua su \( A \);
                \indentitem \item derivabile su \( (a, b) \).
            \end{enumerate}

            Verifico se su \(h\) nell'intervallo \([a,b]\) vale il \textbf{\hyperref[teoRolle]{teorema di Rolle}}:
            \begin{align*}
                h(a)&=\left[f(b) - f(a)\right] \, g(a) - \left[g(b) - g(a)\right] \, f(a)\\
                h(a)&=f(b) \, g(a) - f(a) \, g(a) - f(a) \, g(b) + f(a) \, g(a)\\
                h(a)&=f(b \, )g(a) - f(a) \, g(b)\\
                \\
                h(b)&=\left[f(b) - f(a)\right] \, g(b) - \left[g(b) - g(a)\right] \, f(b)\\
                h(b)&=f(b) \, g(b) - f(a) \, g(b) - f(b) \, g(b) + f(b) \, g(a)\\
                h(b)&=f(b) \, g(a) - f(a) \, g(b)\\
            \end{align*}
            \(h(a)=h(b)\), quindi posso applicare il \textbf{\hyperref[teoRolle]{teorema di Rolle}}, da cui si deriva che \(h\) ha un punto stazionario \(x^*\)

            \[    h'(x) = \left[f(b) - f(a)\right] \, g'(x) - \left[g(b) - g(a)\right] \, f'(x) \]
            
            \[    h'(x^*) = 0 \]

            E quindi infine

            \[  h'(x^*) = 0 \]
            
            \[  \left[f(b) - f(a)\right] \, g'(x^*) - \left[g(b) - g(a)\right] \, f'(x^*) = 0 \]

            \[  \left[f(b) - f(a)\right] \, g'(x^*) = \left[g(b) - g(a)\right] \, f'(x^*) \]
            
            \[  \frac{ f'(x^*) }{ g'(x^*) } = \frac{ f(b) - f(a) }{ g(b) - g(a) } \]
            
            c.v.d.
\end{document}