% Imposto la radice del documento, utile per Visual Studio Code ed altri editor
%! TEX root = ../../dimostrazioni.tex

% Imposto il file come sottofile del documento principale
\documentclass[../dimostrazioni]{subfiles}


\begin{document}

    \chapter{Test di monotonia di \emph{f} su un intervallo aperto}

        \section*{Enunciato}

        \subsection*{Ipotesi}

            Sia \(f(x)\) una funzione tale che
            \begin{align*}
                f : A = (a, b) &\longrightarrow \mathbb{R}\\
                x &\longmapsto y = f(x) 
            \end{align*}

            Supponiamo inoltre che \(f\) sia derivabile su \((a, b)\).

        \subsection*{Tesi}

            \[ f'(x) > 0 \quad \forall \, x \in A \Rightarrow f \] è monotona strettamente crescente su A. 
            \[ f'(x) < 0 \quad \forall \, x \in A \Rightarrow f \] è monotona strettamente decrescente su A.

        \section*{Dimostrazione}

            \medskip

            \subsection*{Caso 1 - \( f'(x) > 0 \quad \forall \, x \in A \)}

            Siano \(x_1, x_2 \in A \mid a < x_1 < x_2 < b \). Seleziono un sottointervallo chiuso interno ad \(A\).
            Su \( [x_1, x_2] \) applico il \textbf{\hyperref[teoLagrange]{teorema di Lagrange}} a \(f\) quindi:
            \[
                \exists \, x_0 \in (x_1, x_2) \mid f(x_2) - f(x_1) = f'(x_0)(x_2 - x_1) 
            \]
            essendo \( f'(x_0) > 0 \) e anche \( x_2 - x_1 > 0 \) ne segue che:
            \[
                \forall \, x_1 < x_2 \Rightarrow f(x_2) > f(x_1)
            \]
            quindi \(f(x)\) è monotona strettamente crescente, c.v.d.

            \subsection*{Caso 2 - \( f'(x) < 0 \quad \forall \, x \in A \)}

            Siano \(x_1, x_2 \in A \mid a < x_1 < x_2 < b \). Seleziono un sottointervallo chiuso interno ad \(A\).
            Su \( [x_1, x_2] \) applico il \textbf{\hyperref[teoLagrange]{teorema di Lagrange}} a \(f\) quindi:
            \[
                \exists \, x_0 \in (x_1, x_2) \, / \, f(x_2) - f(x_1) = f'(x_0)(x_2 - x_1) 
            \]
            essendo \( f'(x_0) < 0 \) e \( x_2 - x_1 > 0 \) ne segue che:
            \[
                \forall \, x_1 < x_2 \Rightarrow f(x_2) < f(x_1)
            \]
            quindi \(f(x)\) è monotona strettamente decrescente, c.v.d.
    
    
\end{document}