% Imposto la radice del documento, utile per Visual Studio Code ed altri editor
%! TEX root = ../../dimostrazioni.tex

% Imposto il file come sottofile del documento principale
\documentclass[../../dimostrazioni]{subfiles}

\begin{document}

    \chapter{Teorema di de l'Hôpital}
    \label{teoHopital}

        \section*{Enunciato}

            \subsection*{Ipotesi}

                Date:
                \begin{align*}
                    f,g : A = [a, b] \longrightarrow \, &\mathbb{R}\\
                    x \longmapsto \, &y = f(x)\\
                                  &y = g(x) 
                \end{align*}

                Supponendo inoltre: 

                \begin{enumerate}
                    \indentitem \item \(f, g\) continue in A e derivabili in \( (a,b) \);
                    \indentitem \item \(f, g\) infinitesime in \(x_0 \in (a,b)\).
                \end{enumerate}

            \subsection*{Tesi}

                \[ 
                    \text{Se } l = \lim_{x \to x_0} \frac{f'(x)}{g'(x)} \text{, allora } l = \lim_{x \to x_0} \frac{f(x)}{g(x)}
                \]

        \section*{Dimostrazione}

            La dimostrazione avviene direttamente utilizzando il \textbf{\hyperref[teoCauchy]{teorema di Cauchy}}:

            \[ \exists \; \vartheta \in (a,b) \Rightarrow \vartheta \in (x_0, x) \]

            Aggiungo \(f(x_0)\) che ricordiamo essere infinitesimo per ipotesi, poi considerando l'intervallo \((x_0, x)\):

            \[  \frac{f(x)}{g(x)} = \frac{f(x) - f(x_0)}{g(x) - g(x_0)} = \frac{f'(\vartheta)}{g'(\vartheta)} \]

            Da cui:
            
            \[  \lim_{x \to x_0} \frac{f(x)}{g(x)} = \lim_{x \to x_0} \frac{f'(\vartheta)}{g'(\vartheta)} = l \]
            
            c.v.d.
\end{document}