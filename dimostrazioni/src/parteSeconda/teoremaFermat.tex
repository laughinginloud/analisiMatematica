% Imposto la radice del documento, utile per Visual Studio Code ed altri editor
%! TEX root = ../../dimostrazioni.tex

% Imposto il file come sottofile del documento principale
\documentclass[../dimostrazioni]{subfiles}

\begin{document}

    \chapter{Teorema di Fermat}
    \label{teoFermat}

        \section*{Definizioni necessarie}

            Si ricordano le seguenti definizioni:

            \begin{itemize}
                \item \(x_0\) è un punto stazionario se \(f'(x_0) = 0\);
                \item \(x_0\) è un punto di ottimo se è un punto di massimo o di minimo locale;
                \item \(x_M\) è un punto di massimo locale se  \(M = f(x_M) \geqslant f(x) \forall  x \in A\) dove M è il valore massimo locale;
                \item \(x_M\) è un punto di minimo locale se  \(m = f(x_m) \leqslant f(x) \forall  x \in A\) dove m è il valore minimo locale.
            \end{itemize}

        \section*{Enunciato}

            \subsection*{Ipotesi}

                Sia \(f(x)\) una funzione tale che
                \begin{align*}
                    f : A = (a, b) &\longrightarrow \mathbb{R}\\
                    x &\longmapsto y = f(x) 
                \end{align*}

                Supponiamo inoltre che:

            \begin{enumerate}
                \indentitem \item \(x_0 \in A\);
                \indentitem \item \(f\) sia derivabile in \(A\);
                \indentitem \item \(x_0\) sia un punto di ottimo.
            \end{enumerate}

            \subsection*{Tesi}

                \[f'(x_0) = 0\] ovvero \(x_0\) è un punto stazionario

        \section*{Dimostrazione}

            \medskip

            \subsection*{Caso 1 - \(x_0\) è un punto di massimo locale}

                \smallskip

                Per l'ipotesi 1 e l'ipotesi 2, quando \(h>0\) possiamo dire che:
                
                \[\frac{f(x_0 + h) - f(x_0)}{h}  \leqslant 0\]

                quando \(h<0\) invece possiamo dire che:

                \[\frac{f(x_0 + h) - f(x_0)}{h}  \geqslant 0\]

                quindi sempre per l'ipotesi di derivabilità valgono le seguenti affermazioni
                
                \[\lim_{x\to 0^+} \frac{f(x_0 + h) - f(x_0)}{h} = L_1 \leqslant 0 \; \text{dove} \; L_1 \, \exists \land L_1 \in \mathbb{R} \]

                \[\lim_{x\to 0^-} \frac{f(x_0 + h) - f(x_0)}{h} = L_2  \geqslant 0 \; \text{dove} \; L_2 \, \exists \land L_2 \in \mathbb{R} \]

                \[L_1 = L_2 = f'(x_0)\]
                
                e quindi
                
                \[0 \leqslant f'(x_0) \leqslant 0\]
                
                da cui
                
                \[f'(x_0)=0\]
                
                c.v.d.

            \subsection*{Caso 2 - \(x_0\) è un punto di minimo locale}

                \smallskip

                Per l'ipotesi 1 e l'ipotesi 2, quando \(h>0\) possiamo dire che:
                
                \[ \frac{f(x_0 + h) - f(x_0)}{h}  \geqslant 0\]

                quando \(h<0\) invece possiamo dire che:

                \[ \frac{f(x_0 + h) - f(x_0)}{h}  \leqslant 0\]

                quindi sempre per l'ipotesi di derivabilità valgono le seguenti affermazioni
                
                \[\lim_{x\to 0^+} \frac{f(x_0 + h) - f(x_0)}{h} = L_1 \geqslant 0 \text{dove} L_1 \exists \land L_1 \in \mathbb{R} \]

                \[\lim_{x\to 0^-} \frac{f(x_0 + h) - f(x_0)}{h} = L_2  \leqslant 0 \text{dove} L_2 \exists \land L_2 \in \mathbb{R} \]

                \[L_1 = L_2 = f'(x_0)\]
                
                e quindi
                
                \[0 \leqslant f'(x_0) \leqslant 0\]
                
                da cui
                
                \[f'(x_0)=0\]
                
                c.v.d.

\end{document}