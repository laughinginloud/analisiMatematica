% Imposto la radice del documento, utile per Visual Studio Code ed altri editor
%! TEX root = ../../dimostrazioni.tex

% Imposto il file come sottofile del documento principale
\documentclass[../../dimostrazioni]{subfiles}

\begin{document}

    \chapter{Teorema di Rolle}
    \label{teoRolle}

        \section*{Enunciato}

            \subsection*{Ipotesi}

                Sia \(f(x)\) una funzione tale che
                \begin{align*}
                    f : A = [a, b] &\longrightarrow \mathbb{R}\\
                    x &\longmapsto y = f(x) 
                \end{align*}

                Supponiamo inoltre che:

                \begin{enumerate}
                    \indentitem \item \(f\) è continua su \(A\) e derivabile su \((a, b)\);
                    \indentitem \item \(f(a) = f(b)\).
                \end{enumerate}

            \subsection*{Tesi}

                \[\exists \, x_0 \in (a,b) \; | \; f'(x_0) = 0 \]

        \section*{Dimostrazione}

            \subsection*{Caso 1 - \(f(x)\) è una funzione costante}

                Il teorema è dimostrato, infatti \(\forall x \in (a,b), \, f'(x) = 0\).

            \subsection*{Caso 2 - \(f(x)\) non è una funzione costante}

                Data la continuità di \(f(x)\) su \(A\) e essendo \(A\) un intervallo chiuso e limitato, vale il \textbf{teorema di Weierstrass}.
                \[ \exists \, M, m \; | \; f(x_m)=m \leqslant f(x) \leqslant f(x_M) = M \quad \forall \, x \in A \]
                e almeno uno tra \(x_m\) e \(x_M\) è interno ad \( (a,b) \), dato che \( m \neq M \), poiché \( f \) non è costante.
                
                Visto che almeno uno dei due punti di ottimo è interno all'intervallo, posso applicare il \textbf{\hyperref[teoFermat]{teorema di Fermat}}, da cui ricavo che il punto di ottimo interno è un punto stazionario e quindi:
                \[\exists \, x_0 \in (a,b) \; | \; f'(x_0) = 0 \]
                c.v.d.

    
\end{document}