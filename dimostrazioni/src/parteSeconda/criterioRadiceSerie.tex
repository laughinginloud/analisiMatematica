% Imposto la radice del documento, utile per Visual Studio Code ed altri editor
%! TEX root = ../../dimostrazioni.tex

% Imposto il file come sottofile del documento principale
\documentclass[../../dimostrazioni]{subfiles}

\begin{document}

    \chapter{Criterio della radice per la convergenza delle serie a termini positivi}
    \label{criterioRadiceSerie}

        \section*{Enunciato}

            Sia \(\sum a_n\) una serie a termini positivi, con \(a_n > 0, \, \forall \, n\). Se
             \[\sqrt[n]{a_n} \rightarrow l \qquad \text{per} \; n  \rightarrow +\infty \]
            
            Allora
            \[
                l \;
                \begin{cases}
                    \text{diverge} & \text{se} \; l > 1\\
                    \text{il criterio non si applica} & \text{se} \; l = 1\\
                    \text{converge} & \text{se} \; 0 \leqslant l < 1
                \end{cases}
            \]
            
        \section*{Dimostrazione}
            
            \subsection*{Caso 1 - \(0 \leqslant l < 1\)} 
                    
                Introduco una \textbf{successione ausiliaria}
                \begin{gather*}
                    b_n = \sqrt[n]{a_n}\\
                    \lim_{n \to +\infty} b_n = l \qquad \text{e so che} \; l < 1
                \end{gather*}

                Per la definizione di limite
                \[\forall \, \mathrm{B}_\varepsilon (l), \, \exists \, M \mid \forall \, n > M, \, b_n \in \mathrm{B}_\varepsilon (l) \]

                Scegliamo \(\varepsilon\) in modo che \(\varepsilon < 1 - l\) da \(M\) in poi. Dunque,
                \begin{align*}
                    l - \varepsilon < b_n &< l + \varepsilon \qquad (<1) \\
                    \sqrt[n]{a_n} &< l + \varepsilon \qquad (<1) \\
                    a_n &< (l+\varepsilon)^{n}  
                \end{align*}
                                     
                Applico il \textbf{\hyperref[serieconfronto]{criterio del confronto}} tra \(\sum a_n\) e \(\sum(l + \varepsilon)^n\),
                dove \(\sum(l + \varepsilon)^n\) è la geometrica di ragione \(q = l + \varepsilon\), con \(-1 < q < 1\).
                Essendo quest'ultima convergente, possiamo concludere che anche la serie di partenza \textbf{converge}.

            \subsection*{Caso 2 - \(l > 1\)}

                Definiamo una successione ausiliaria \(b_n\) come
                \[ b_n = \sqrt[n]{a_n} \]

                Sappiamo inoltre che
                \[ \lim_{n \to +\infty}b_n = l > 1 \]

                Da ciò possiamo dedurre che
                \[\forall \, n , \, \sqrt[n]{a_n} = 1 + k\]
                con \(k > 0\).

                Questo ci permette di dividere la serie di partenza in una somma di due serie distinte:
                \[\sum a_n = \sum 1 + \sum k\]

                Ma, poiché \(\sum 1\) diverge a \(+ \infty\) e \(\sum k > 0\), perciò sicuramente non diverge a \(- \infty\),
                sicuramente la serie risultante dalla loro somma, ovvero la serie di partenza, diverge a \(+ \infty\), c.v.d.

\end{document}