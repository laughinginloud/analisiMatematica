% Imposto la radice del documento, utile per Visual Studio Code ed altri editor
%! TEX root = ../../dimostrazioni.tex

% Imposto il file come sottofile del documento principale
\documentclass[../dimostrazioni]{subfiles}

\begin{document}

    \chapter{Criterio della radice per la convergenza delle serie a termini positivi}
    \label{criterioRadiceSerie}

        \section*{Enunciato}

            Sia \(\sum_{n\dots}^{+\infty}\) \(a_n\) una serie a termini positivi \(a_n\) > 0 \(\forall n\). Se:
             \[\sqrt[n]{a_n} \rightarrow l \qquad  per \; n  \rightarrow +\infty \]
            
            Allora:
            \[
                l \;
                \begin{cases}
                    \text{diverge} & \text{se} \; l > 1\\
                    \text{il criterio non si applica} & \text{se} \; l = 1\\
                    \text{converge} & \text{se} \; 0 \leqslant l < 1
                \end{cases}
            \]
            
        \section*{Dimostrazione}
            
            \subsection*{Caso 1 - \(0 \leqslant l < 1\)} 
                    
                Introduco una \textbf{successione ausiliaria}
                \[b_n = \sqrt[n]{a_n} \]  
                \[\lim_{n \to +\infty} b_n = l \qquad \text{e so che} \, l < 1\]

                Per la definizione di limite:
                \[\forall \, B_\varepsilon (l) \; \exists \, M \mid \forall \, n > M \quad b_n \in B_\varepsilon (l) \]

                Scelgo \(\varepsilon\) in modo che \(\varepsilon < 1 -l \)
                \begin{align*}
                    l - \varepsilon < b_n &< l + \varepsilon \qquad (<1) \\
                    \sqrt[n]{a_n} &< l + \varepsilon \qquad (<1) \\
                    a_n &< (l+\varepsilon)^{n}  
                \end{align*}
                                     
                Applico il \textbf{\hyperref[serieconfronto]{criterio del confronto}} tra \(\sum a_n\) e \(\sum(l + \varepsilon)\),
                dove \(\sum(l + \varepsilon)\) è la geometrica di ragione \(q = l + \varepsilon\), con \(-1 < q < 1\).
                Essendo quest'ultima convergente, possiamo concludere che anche la serie di partenza \textbf{converge}.

            \subsection*{Caso 2 - \(l > 1\)}

                %TODO
\end{document}