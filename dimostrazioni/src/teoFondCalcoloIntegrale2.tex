% Imposto la radice del documento, utile per Visual Studio Code ed altri editor
%! TEX root = ../dimostrazioni.tex

% Imposto il file come sottofile del documento principale
\documentclass[../dimostrazioni]{subfiles}

\begin{document}

    \chapter{Secondo Teorema Fondamentale del Calcolo Integrale}
    \label{teoFondCalcoloIntegrale2}

    Ha due formulazioni:

    \section*{Prima Forma}

        Data una funzione limitata e Riemann-integrabile:
        \begin{align*}
            f : A = [a, b] &\longrightarrow \mathbb{R}\\
            t &\longmapsto y = f(t) 
        \end{align*}

        È detta \textbf{funzione integrale} la funzione \(G\):
        \begin{align*}
            G(x) = \int_{a}^{x} f(t) dt \qquad   &{G : [a, b] \longmapsto \mathbb{R}} \\
            &{x \longmapsto G(x) = \int_{a}^{x} f(t) dt}
        \end{align*}

        \(G\) è una funzione \textbf{continua}.
            
            \subsection*{Dimostrazione}

                Voglio dimostrare che 

                \[\forall x_0 \in [a, b] \qquad G(x_0) = \lim_{x \to  x_0} G(x) \]
                
                Per comodità consideriamo il limite da destra e \(a < x_0 < x < b \)
                \begin{align*}
                    \lim_{x \to {x_0}^{+}}G(x) =& \lim_{x to x_0}{\int_{a}^{x}f(t) dt} = \\
                    =& \lim_{x \to x_0} \left[ \int_{a}^{x_0} + \int_{x_0}^{x} \right] = \\
                    =& \lim_{x \to x_0} \left[ G(x_0) + \int_{x_0}^{x} f(t) dt \right] = \\
                    =& G(x_0)
                \end{align*}

                \[m \leqslant f(t) \leqslant M \qquad \text {accumulo tra} \qquad x_0 \; ed \; x \]

                \[m(x-x_0) \, \leqslant \int_{x_0}^{x} f(t) dt \, \leqslant M(x_0-x) \]

                L'integrale definito è \emph{infinitesimo} perché limitato tra quantità che tendono a 0.




    \section*{Seconda Forma}

        Data una funzione continua:
        \begin{align*}
            f : A = [a, b] &\longrightarrow \mathbb{R}\\
            t &\longmapsto y = f(t) 
        \end{align*}

        È detta \textbf{funzione integrale} la funzione \(G\):
        \begin{align*}
            G(x) = \int_{a}^{x} f(t) dt \qquad   &{G : [a, b] \longmapsto \mathbb{R}} \\
            &{x \longmapsto G(x) = \int_{a}^{x} f(t) dt}
        \end{align*}

        \(G\) è una funzione \textbf{derivabile}.

        \[G \in C^{1}([a, b]) \quad e \quad G'(x) = f(x) \qquad \forall x \in [a,b] \]
        
            \subsection*{Dimostrazione}

                Sia \(x_0 \in (a,b) \), vogliamo dimostrare che \(G\) è derivabile in \(x_0\)

                \begin{align*}
                    \frac{G(x_0+h) - G(x_0)}{h} =& \frac{1}{h} \left[ \int_{a}^{x_0+h} f(t) dt - \int_{a}^{x-0} f(t)dt \right] \\
                    =& \frac{1}{h} \int_{x_0}^{x_0+h}f(t) dt & \text{VMI dif su} [x_0, {x_0} +h]\\
                    =& f(\theta) \longmapsto f(x_0)  & \text{per la seconda proprietà del VMI}& \\
                    &                                &  con \; h \rightarrow 0 \; e \; \theta \rightarrow x_0&
                \end{align*}

\end{document}