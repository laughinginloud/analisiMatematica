% Imposto la radice del documento, utile per Visual Studio Code ed altri editor
%! TEX root = ../dimostrazioni.tex

% Imposto il file come sottofile del documento principale
\documentclass[../dimostrazioni]{subfiles}

\begin{document}

    \chapter{Secondo Teorema Fondamentale del Calcolo Integrale}
    \label{teoFondCalcoloIntegrale2}

        \section*{Definizioni necessarie}

            Si ricorda che è detta \textbf{funzione integrale} la funzione \(G\):
            \begin{align*}
                G(x) = \int_{a}^{x} f(t) dt \qquad   &{G : [a, b] \longmapsto \mathbb{R}} \\
                &{x \longmapsto G(x) = \int_{a}^{x} f(t) dt}
            \end{align*}

        \section*{Prima Forma}

            \subsection*{Enunciato}

                \subsubsection*{Ipotesi}
        
                    Data una funzione \textbf{limitata e Riemann-integrabile}:
                        \begin{align*}
                            f : A = [a, b] &\longrightarrow \mathbb{R}\\
                            t &\longmapsto y = f(t) 
                        \end{align*}

                \subsubsection*{Tesi}

                    \(G\) è una funzione \textbf{continua}.
            
            \subsection*{Dimostrazione}

                Voglio dimostrare che 

                \[\forall x_0 \in [a, b] \qquad G(x_0) = \lim_{x \to  x_0} G(x) \]
            
                \subsubsection*{Caso 1 - \(a < x_0 < x < b \)}
                    Consideriamo quindi il limite da destra:
                    \begin{align*}
                        \lim_{x \to {x_0}^{+}}G(x) =& \lim_{x to x_0}{\int_{a}^{x}f(t) dt} = \\
                        =& \lim_{x \to x_0} \left[ \int_{a}^{x_0} + \int_{x_0}^{x} \right] = \\
                        =& \lim_{x \to x_0} \left[ G(x_0) + \int_{x_0}^{x} f(t) dt \right]
                    \end{align*}

                    Se \( \lim_{x \to {x_0}^{+}} \int_{x_0}^{x} f(t) dt \) fosse \emph{infinitesimo} allora:

                    \[  \lim_{x \to {x_0}^{+}}G(x) = G(x_0) \]

                    che dimostrerebbe la continuità di \(G(x)\). Passiamo quindi a dimostrare che \( \lim_{x \to {x_0}^{+}} \int_{x_0}^{x} f(t) dt \) è \emph{infinitesimo}:

                    \[m \leqslant f(t) \leqslant M \qquad \text {accumulo tra} \qquad x_0 \; ed \; x \]

                    \[m(x-x_0) \, \leqslant \int_{x_0}^{x} f(t) dt \, \leqslant M(x-x_0) \]

                    L'integrale definito è \emph{infinitesimo} perché limitato tra quantità che tendono a 0. Questo dimostra la continuità di \(G(x)\).

                    c.v.d.
                
                \subsubsection*{Caso 2 - \(a < x < x_0 < b \)}
                    Consideriamo quindi il limite da sinistra:
                    \begin{align*}
                        \lim_{x \to {x_0}^{-}}G(x) =& \lim_{x to x_0}{\int_{a}^{x}f(t) dt} = \\
                        =& \lim_{x \to {x_0}^{-}} \left[ \int_{a}^{x_0}f(t) dt - \int_{x}^{x_0}f(t) dt \right] = \\
                        =& \lim_{x \to {x_0}^{-}} \left[ G(x_0) - \int_{x}^{x_0} f(t) dt \right]
                    \end{align*}

                    Se \( \lim_{x \to {x_0}^{-}} \int_{x}^{x_0} f(t) dt \) fosse \emph{infinitesimo} allora:

                    \[  \lim_{x \to {x_0}^{-}}G(x) = G(x_0) \]

                    che dimostrerebbe la continuità di \(G(x)\). Passiamo quindi a dimostrare che \( \lim_{x \to {x_0}^{-}} \int_{x}^{x_0} f(t) dt \) è \emph{infinitesimo}:

                    \[m \leqslant f(t) \leqslant M \qquad \text {accumulo tra} \qquad x \; ed \; x_0 \]

                    \[m(x_0-x) \, \leqslant \int_{x}^{x_0} f(t) dt \, \leqslant M(x_0-x) \]

                    L'integrale definito è \emph{infinitesimo} perché limitato tra quantità che tendono a 0. Questo dimostra la continuità di \(G(x)\).

                    c.v.d.
            
            \section*{Seconda Forma}

                \subsection*{Enunciato}

                    \subsubsection*{Ipotesi}

                        Data una funzione \textbf{continua}:
                        \begin{align*}
                            f : A = [a, b] &\longrightarrow \mathbb{R}\\
                            t &\longmapsto y = f(t) 
                        \end{align*}

                    \subsubsection*{Tesi}
                    
                        \(G\) è una funzione \textbf{derivabile}.

                        \[G \in C^{1}([a, b]) \quad e \quad G'(x) = f(x) \qquad \forall x \in [a,b] \]
        
                    \subsection*{Dimostrazione}

                        Sia \(x_0 \in (a,b) \), vogliamo dimostrare che \(G\) è derivabile in \(x_0\)


                        \subsubsection*{Caso 1 - \(h>0\)}

                            \begin{align*}
                                \frac{G(x_0+h) - G(x_0)}{h} =& \frac{1}{h} \left[ \int_{a}^{x_0+h} f(t) dt - \int_{a}^{x_0} f(t)dt \right] \\
                                                            =& \frac{1}{h} \int_{x_0}^{x_0+h}f(t) dt & \text{VMI dif su} [x_0, {x_0} +h]\\
                                \exists \, \theta \in (x_0, x_0+h) | =& f(\theta) \longmapsto f(x_0)          & \text{per la seconda proprietà del VMI}& \\
                                &                            &  con \; h \rightarrow 0^{+}&
                            \end{align*}
                            
                            Dimostrando che non solo \(G(x)\) è derivabile su \((a,b)\) data l'arbitrarietà di \(x_0\), 
                            ma anche che la derivata di \(G(x)\) è \(f(x)\). c.v.d.

                        \subsubsection*{Caso 2 - \(h<0\)}

                            \begin{align*}
                                \frac{G(x_0+h) - G(x_0)}{h} =& \frac{1}{h} \left[ \int_{a}^{x_0+h} f(t) dt - \int_{a}^{x_0} f(t)dt \right] \\
                                                            =& \frac{1}{h} \left[ \int_{a}^{x_0+h} f(t) dt - \int_{a}^{x_0+h} f(t)dt -\int_{x_0+h}^{x_0} f(t)dt \right] \\
                                                            =& \frac{1}{-h} \int_{x_0+h}^{x_0}f(t) dt & \text{VMI dif su} [x_0 + h, x_0]\\
                                \exists \, \theta \in (x_0+h, x_0) | =& f(\theta) \longmapsto f(x_0)          & \text{per la seconda proprietà del VMI}& \\
                                &                            &  con \; h \rightarrow 0^{-}&
                            \end{align*}
                            
                            Dimostrando che non solo \(G(x)\) è derivabile su \((a,b)\) data l'arbitrarietà di \(x_0\), 
                            ma anche che la derivata di \(G(x)\) è \(f(x)\). c.v.d.            
\end{document}