% Imposto la radice del documento, utile per Visual Studio Code ed altri editor
% !TEX root = ../dimostrazioni.tex

% Utilizzo il tipo libro impostato per evitare pagine vuote, utili solo per la stampa
\documentclass[oneside]{book}

\pagestyle{plain}

% Utilizzo il pacchetto per le traduzioni ed imposto la lingua
\usepackage{polyglossia}
\setmainlanguage{italian}

% Imposto il margine delle pagine
\usepackage[margin=1in]{geometry}

% Importo i pacchetti per le immagini con descrizioni, che torneranno utili per le dimostrazioni conteneti disegni
\usepackage{graphicx}
\usepackage{subcaption}

% Importo i pacchetti per la matematica
\usepackage{mathtools}
\usepackage{unicode-math}
\usepackage{amssymb}

% Imposto i collegamenti
\usepackage[unicode=true,bookmarks=true,bookmarksnumbered=false,bookmarksopen=false,breaklinks=false,pdfborder={0 0 0},backref=section,colorlinks=false]{hyperref}

% Importo il pacchetto per la personalizzazione del titolo e lo modifico
\usepackage{titlesec}
\titleformat{\chapter}[display]{\normalfont\LARGE\bfseries}{Dimostrazione numero \thechapter}{.5em}{\Huge}

% Importo il pacchetto per i sottofile
\usepackage{subfiles}

% Aggiunto un comando per l'indentazione degli elementi di un elenco
\newcommand{\indentitem}{\setlength\itemindent{25pt}}

% Imposto le informazioni presenti sulla copertina
\author{Giovanni Manfredi e Mattia Martelli}
\date{}
\title{Dimostrazioni di Analisi matematica 1}