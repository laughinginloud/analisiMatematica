% Imposto la radice del documento, utile per Visual Studio Code ed altri editor
%! TEX root = ../../analisi1.tex

% Imposto il file come sottofile del documento principale
\documentclass[../../analisi1]{subfiles}

\begin{document}

    \chapter{Teorema del resto secondo Lagrange}
    \label{teorestoLagrange}

        \section*{Definizioni necessarie}

            Si ricorda che il \textbf{polinomio di Taylor (\(T _n ^ f (x) \))} è così definito:
            \[ \sum_{k = 0}^{n} \frac{f^{(k)} (x_0)}{k!}(x-x_0)^k\]

        \section*{Enunciato}

            \subsection*{Ipotesi}

                Sia \(f(x)\) una funzione tale che
                \begin{align*}
                    f : A = (a, b) &\longrightarrow \mathbb{R}\\
                    x &\longmapsto y = f(x) 
                \end{align*}

                Supponiamo inoltre che:

                \begin{enumerate}
                    \indentitem \item \(f \in C^{n+1} (A) \);
                    \indentitem \item \(x_0 \in A\).
                \end{enumerate}

            \subsection*{Tesi}

                \[  \exists \, \vartheta \in (x_0, x) \; | \; f(x) - T _n ^ f (x) = \frac{f^{n+1}(\vartheta)}{(n+1)!}(x-x_0)^{n+1} \]

        \section*{Dimostrazione}

            \medskip

            Considero due \textbf{funzioni ausiliarie \(g(x), w(x)\)} così definite:
            \begin{alignat*}{2}
                g(x) &= f(x) - T_n^f (x) \qquad g(x) &&\in C^{n+1} (A)\\
                w(x) &= (x - x_0)^{n+1} \qquad \; w(x) &&\in C^{\infty} (A)
            \end{alignat*}

            \newpage

            Calcolo \( g(x_0)\, , \, g'(x_0)\, ,\, \dots\, , \, g ^{(n+1)}(x_0)  \):
            \begin{align*}
                g(x_0) &= f(x_0) - \bigg[\frac{f(x_0)}{0!}1 + \frac{f'(x_0)}{1!}(x_0-x_0) + \frac{f''(x_0)}{2!}(x_0-x_0)^2 + \dots + \frac{f^{(n)}(x_0)}{n!}(x_0-x_0)^n \bigg] = 0\\
                g'(x_0) &= f'(x_0) - \bigg[\frac{f'(x_0)}{1!}1 + \frac{f''(x_0)}{2!}2(x_0-x_0) + \dots + \frac{f^{(n)}(x_0)}{n!}n(x_0-x_0)^{n-1} \bigg] = 0\\
                g''(x_0) &= 0\\
                &\dots\\
                g^{(n)} (x_0) &= 0\\
                g^{(n+1)} (x_0) &= f^{(n+1)}(x_0) - 0 = f^{(n+1)}(x_0)
            \end{align*}

            \bigskip
            
            Calcolo \( w(x_0)\, , \, w'(x_0)\, ,\, \dots\, , \, w ^{(n+1)}(x_0)  \):
            \begin{align*}
                w(x_0) &= (x_0 - x_0)^{n+1} = 0\\
                w'(x_0) &= (n+1)(x_0 - x_0)^{n} = 0\\
                w'(x_0) &= (n+1)(n)(x_0 - x_0)^{n-1} = 0\\
                & \dots\\
                w^{(n)} (x_0) &= \big[(n+1)!\big](x_0 - x_0) = 0\\
                w^{(n+1)} (x_0) &= \big[(n+1)!\big] 1 = (n+1)!
            \end{align*}

            \bigskip

            Toniamo ora su ciò che dobbiamo dimostrare:
            \begin{align*}
                \exists \, \vartheta \in (x_0, x) \; | \; f(x) - T _n ^ f (x) &= \frac{f^{n+1}(\vartheta)}{(n+1)!}(x-x_0)^{n+1}\\
                \frac{f(x) - T _n ^ f (x)}{(x-x_0)^{n+1}} &= \frac{f^{n+1}(\vartheta)}{(n+1)!}
            \end{align*}

            Notiamo che \(  \frac{f(x) - T _n ^ f (x)}{(x-x_0)^{n+1}} = \frac{g(x)}{w(x)}    \) quindi utilizzando il \textbf{\hyperref[teoCauchy]{teorema di Cauchy}}:
            \begin{align*}
                \frac{g(x)}{w(x)} &= \frac{g(x) - g(x_0)}{w(x) - w(x_0)}\\
                \exists \, x_1 \in (x_0, x) \qquad &= \frac{g'(x_1)}{w'(x_1)} = \frac{g'(x_1) - g'(x_0)}{w'(x_1) - w'(x_0)}\\
                \exists \, x_2 \in (x_0, x_1) \qquad &= \frac{g''(x_2)}{w''(x_2)} = \frac{g''(x_2) - g''(x_0)}{w''(x_2) - w''(x_0)}\\
                \exists \, x_3 \in (x_0, x_2) \qquad &= \frac{g'''(x_3)}{w'''(x_3)} = \, \dots\\
                \exists \, \vartheta \in (x_0, x_n) \qquad &= \frac{g^{(n+1)}(\vartheta)}{w^{(n+1)}(\vartheta)} \qquad \text{\emph{Iterando \(n\) volte}}
            \end{align*}

            Notiamo anche che possiamo fare questo perché da come abbiamo dimostrato prima calcolandolo,\\
            \(g(x_0)\,,\, g'(x_0)\,,\, \dots\,,\, g^{(n)} (x_0) \) e \( w(x_0)\,,\, w'(x_0)\,,\, \dots\,,\, w^{(n)} (x_0)\) sono infinitesimi.\\

            \newpage
            
            Quindi le derivate (\(n+1\))-esime dal precedente calcolo di \(g(x)\) e \(w(x)\) sono:
            \[  \frac{g^{(n+1)}(\vartheta)}{w^{(n+1)}(\vartheta)} = \frac{f^{n+1}(\vartheta)}{(n+1)!} \]

            Quindi per come abbiamo definito \(g(x)\) e \(w(x)\):
            \[  \frac{f(x) - T _n ^ f (x)}{(x-x_0)^{n+1}} = \frac{g(x)}{w(x)} = \frac{g^{(n+1)}(\vartheta)}{w^{(n+1)}(\vartheta)} = \frac{f^{n+1}(\vartheta)}{(n+1)!}   \]

            Da cui:
            \begin{align*}
                \frac{f(x) - T _n ^ f (x)}{(x-x_0)^{n+1}} &= \frac{f^{n+1}(\vartheta)}{(n+1)!}\\
                f(x) - T _n ^ f (x) &= \frac{f^{n+1}(\vartheta)}{(n+1)!}(x-x_0)^{n+1}
            \end{align*}

            c.v.d.
        \end{document}