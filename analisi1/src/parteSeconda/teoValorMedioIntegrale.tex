% Imposto la radice del documento, utile per Visual Studio Code ed altri editor
%! TEX root = ../../analisi1.tex

% Imposto il file come sottofile del documento principale
\documentclass[../../analisi1]{subfiles}

\begin{document}

    \chapter{Teorema del valor medio integrale}
    \label{teoValorMedioIntegrale}

    \section*{Enunciato}

        \subsection*{Ipotesi}

            Sia \(f(x)\) una funzione limitata tale che
            \begin{align*}
                f : A = [a, b] &\longrightarrow \mathbb{R}\\
                t &\longmapsto y = f(t) 
            \end{align*}

            Supponiamo inoltre che:

                \begin{enumerate}
                    \indentitem \item \(m\) = min \(f\) su \([a, b]\);
                    \indentitem \item \(M\) = max \(f\) su \([a, b]\).
                \end{enumerate}  

        \subsection*{Definizione}

            \[ \frac{1}{b-a} \int_{a}^{b} \! f(t) \, \mathrm{d} t \]
            purché \(f\) sia Riemann-integrabile.

    \section*{Proprietà 1}

        \[ m \leqslant VMI \leqslant M \]

        \subsection*{Dimostrazione}
    
            \[m \leqslant f(t) \leqslant M  \qquad \forall t \in [a, b] \]

            Integrale definito:
            \[\int_{a}^{b} \! m \, \mathrm{d} t \leqslant \int_{a}^{b} \! f(t) \, \mathrm{d} t \leqslant \int_{a}^{b} \! M \, \mathrm{d} t\]

            Per la monotonia:
            \begin{alignat*}{2}
                m(b-a) &\leqslant \quad \: \int_{a}^{b} \! f(t) \, \mathrm{d} t &&\leqslant M(b-a) \\
                m &\leqslant \frac{1}{b-a} \int_{a}^{b} \! f(t) \, \mathrm{d} t &&\leqslant M
            \end{alignat*}

    \section*{Proprietà 2}

        Se \(f \in C^0 ([a,b]) \) allora:
        \[\exists \, \vartheta\in [a, b] \mid f(\vartheta) = VMI \]

        \subsection*{Dimostrazione}
    
            Valendo Weierstrass e \hyperref[teoDarboux]{Darboux}:
            \[m \leqslant VMI \leqslant M \]

\end{document}