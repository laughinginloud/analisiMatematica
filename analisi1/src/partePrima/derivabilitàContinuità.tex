% Imposto la radice del documento, utile per Visual Studio Code ed altri editor
%! TEX root = ../../analisi1.tex

% Imposto il file come sottofile del documento principale
\documentclass[../../analisi1]{subfiles}

\begin{document}

    \chapter{Derivabilità implica continuità}

        \section*{Enunciato}

            Se \(f\) è derivabile in un punto \(x_0\) allora \(f\) è continua in \(x_0\).

        \section*{Dimostrazione}

            Per la definizione di derivata, se la funzione è derivabile in un punto \(x_0\)
            significa che
            \begin{align*}
                m = &\lim_{h \to 0} \frac{f(x_0 + h) - f(x_0)}{h} \quad \exists \text{ finito}\\
                &\lim_{h \to 0} \frac{f(x_0 + h) - f(x_0)}{m h} = 1
            \end{align*}

            Dunque, per la definizione di asintotico, con \(h \to 0\),
            \begin{align*}
                f(x_0 + h) - f(x_0) &\sim m h\\
                f(x_0 + h) - f(x_0) &= m h + o(m h)\\
                f(x_0 + h) &= f(x_0) + m h + o(m h)
            \end{align*}

            Bisogna perciò calcolare \(\lim_{h \to 0} f(x_0 + h)\). Eseguendo la sostituzione
            \(x = x_0 + h\) si ottiene
            \[
                \lim_{h \to 0} f(x) = \lim_{h \to 0} f(x_0) + \underbrace{mh}_{\to 0} + \underbrace{o(mh)}_{\to 0} = f(x_0)
            \]

            c.v.d.

\end{document}